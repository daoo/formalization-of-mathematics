\subsection{Coq}
\subsubsection{Vad är Coq}
Coq är en interaktiv theorembevisare som innehåller ett beroendetypat
funktionellt programmeringsspråk. Coq används för formalisera matematik och för
att verifiera program. Det är viktigt att påpeka att Coq inte är en automatisk
bevishanterare vilket innebär att Coq löser ingenting automatiskt utan det är
upp till användaren att bevisa och verifiera sina bevis.

\subsubsection{Coq's Historia}
Skrivet av Anders

\subsubsection{Tillämpningar}

\begin{itemize}

\item Fyrfärgssatsen\autocite{gonthier2008formal} säger att, givet varje möjlig
uppdelning av ett plan i sammanhängande regioner, så krävs det högst fyra
färger för att färglägga alla regionerna så att inga angränsande regioner har
samma färg. Två regioner anses vara angränsande om de delar en gemensam kant
som inte är ett hörn.

Under årens lopp har det förekommit många olika bevis och motbevis som har
bevisats vara falska och  det var först 1976 som det första korrekta beviset
kom. Stora delar av beviset var då gjorda i en dator och det gick inte att
bekräfta att dessa delar var korrekta. För att bevisa att datordelarna av
beviset var korrekta så formaliserades beviset i Coq och detta ledde till
utvecklandet av SSReflect.

\item CompCert\autocite{compcert} är ett projekt som utforskar möjligheten att
utveckla formellt bevisade kompilatorer. Anledningen att man vill ha en
formellt bevisad kompilator är att vid vissa optimeringar så kan kompilatorn
skapa buggar och beräkningsfel. Att kompilatorn är formellt bevisad innebär att
det finns ett matematiskt bevis, som kan kontrolleras genom en mekanisk check,
för att den exekverbara koden beter sig så som står föreskrivet i källkoden.
Rent konkret innebär detta att man är garanterad att den exekverbara koden inte
innehåller buggar som är skapade av kompilatorn. Detta projektet resulterade i
en fungerande C-kompilator som stödjer hela ANSI C med ett få undantag och som
faktiskt är snabbare än GCC utan några optimeringar.

\item Feit-Thompsons är en sats inom matematisk gruppteori som säger att en
ändlig grupp alltid är lösbar om dess ordning är
udda\cite{aschbacher2004status}. Denna sats bevisades av Walter Feit och John
Griggs Thompson 1963. Beviset för Feit-Thompsons sats är stort och sträcker sig
över två volymer, det är mycket material för en person att sätta sig in i och
verifiera för hand. Storleken på beviset och därmed möjligheten till någon dold
miss i beviset är en av anledningarna till varför beviset för Feit-Thompsons
sats är intressant att formalisera i Coq. I samband med formaliseringen av
beviset så har även en stor del av matematisk gruppteori verifierats.

\end{itemize}

\subsubsection{Logik (Konstruktiv)}
Tror att det är bättre om någon från matte gör det här

\subsubsection{Curry-Howard isomorphism}
Enligt Curry Howard isomorpismen så är propositioner samma sak som typer och
bevis är samma sak som program. Om vi tar en närmare titt på
funktionsdefinitionen $a \rightarrow b$ så kan vi tolka det som att givet ett
bevis för a så får vi ett bevis för b.
Enligt Curry Howard isomorpismen så är propositioner samma sak som typer och
bevis är samma sak som program. Om vi tar en närmare titt på
funktionsdefinitionen $a \rightarrow b$ så kan vi tolka det som att givet ett
bevis för a så får vi ett bevis för b.
\begin{align*}
Propotioner &= Typer \\
Bevis &= Programs
\end{align*}

\subsubsection{Polymorphism}
En enkel förklaring till polymorphism är att en funktion kan appliceras på
flera olika typer av parametrar.
En lite mer komplicerad förklaring är att en polymorphisk funktion är uppbygd av
två olika lambda uttryck där det första har typer som parametrar och det andra
uttrycket har termer som beror av de angivna typerna som parametrar.
\begin{equation}
  \lambda_{Typer} \rightarrow (\lambda_{Termer} \rightarrow x)
  \label{polymorphsk funktion}
\end{equation}
Den yttre $\lambda$-funktionen tar en eller flera typer som parametrar och ger
tillbaks en ny $\lambda$-funktion som nu har sina parametrar bundna till de
typer som angavs i den yttre $\lambda$-funktionen.

Som de flesta funktionella programmeringspråk så har Coq stöd för ad hoc
polymorphism som även kallas överlagring. Detta innebär att typerna i
funktionen bestämms av sammanhanget. Då coq är ett strikt typat språk används
typklasser för att beskriva vilka typer som är tillåtna.
Även då Coq har fullt stöd för typklasser och överlagring så används detta inte
så mycket utan istället används kanoniska strukturer {\it (Eng Canonical
Structures)}. Kanoniska strukturer är för komplicerade för att ingå i detta
kandidatarbete men de är värda att nämnas då de ofta används för att lösa
polymorphism i Coq.

I följande exempel defineras en lista som kan innehålla värden av alla möjliga
typer. Den undre definitionen i exemplet är en rekursiv funktion som lägger till
ett ellement i en lista. Denna funktionen är polymorphisk och fungerar alltså på
alla listor oberoende vilken typ på värden som listan innehåller.
\begin{verbatim}
Inductive list (X:Type) : Type :=
  | nil  : list X
  | cons : X -> list X -> list X .

Fixpoint append (X:Type) (l : list X) (x : X) : list X :=
  match l with
  | nil       => cons X x (nil X)
  | cons a l' => cons X a (append l' x)
  end.
\end{verbatim}

\subsubsection{Beroendetyper}
I en polymorphisk funktion så kan en parametertyp bero på vilka typer de
tidigare parametrarna har haft. I beroendetypning så går vi ett steg längre och
låter typerna bero på värdet av en tidigare parameter
Funktionsdefinition för en funktion som gör om en lista till en vector. En
vector är en lista med en fix längd och när vi gör om en lista till en vector
så vill vi alltså att den resullterande vectorn har samma längd som listan.
\begin{verbatim}
toVec: {a : Type} (list : List a) : Vec a (lenght list)
\end{verbatim}

Det är inte bara funktionella språk som det finns beroendetypade funktioner.
Ett exempel på beroendetypning som de flesta antagligen är bekanta med är
\texttt{printf} i C. I funktionen \texttt{printf} beror antalet parametrar och
deras typer på värdet av den första parametern.

\begin{verbatim}
printf("%s is %d years old and %f.1cm long", name ,age , lenght)
\end{verbatim}

\begin{comment}
CoqArt
http://mattam.org/research/publications/Programming_with_Dependent_Types_in_Coq-PPS-260209.pdf
** DONE Grund och taktikspråk
   CLOSED: [2013-05-10 fre 10:41]
\end{comment}

\subsubsection{Grund och taktikspråk}
Coq består av två olika delspråk. Grundspråket kallas Gallina och liknar till
viss del OCaml. Det är i Gallina som de definitioner och funktioner som ska
bevisas skrivs.

Coq innehåller också ett taktikspråk som heter Ltac och innehåller olika
taktiker för att påverka de hypoteser och mål som ska bevisas. Ltac gör det
möjligt att använda samma metoder i Coq som man använder när man skapar ett
pappersbevis.
Coq är en interaktiv teorembevisare så när en viss taktik används uppdateras de
hypoteser och mål som ska bevisas och användaren anger då en ny taktik och
detta fortsätter tills alla målen är lösta.

\subsubsection{Extrahering av program}
Det inte går att köra program i Coq utan bara att bevisa korrektheten hos dem.
Däremot så går det att exportera program skrivna Coq till OCaml, Haskell eller
Scheme. Programmen går tyvärr inte att köra dirrekt effter extraktionen utan
den generarade koden kan behövas städas upp och viss form av initialisering kan
behövas.

\subsubsection{Alternativ till Coq}
Det finns många olika program och språk för att formallisera och bevisa olika
former av matematik eller logik. En sak som de flesta har gemensamt är att de
är uppbyggd på funktionella programmerings paradigmer.

\begin{itemize}
\item Agda är utvecklat på chalmers och påminner till stor del om Haskell. Till
skillnad mot Coq så finns det inga inbyggda taktiker.  Agda är inte lika
matematiskt inriktad som Coq utan används mer till att bevisa korrekthet hos
program. En fördel med Agda är att alla Unicode tecken är tillåtna vilket gör
det enkelt att skriva sina program och bevis på samma sätt som man skulle göra
det på paper.

\item Z3 är ett språk som har utvecklats av Microsoft för att förenkla och bevisa
olika theorem. Kan användas tillsammans med flera stora ickefunktionella språk
som Python,C och .NET.

\item HOL, Högre Ordningens Logik är en av de första interaktiva teorem bevisarna och
HOL-light som är en vidareutveckling av det används idag av Intel för att
bevisa att vissa hårdvarukomponenter fungerar korrekt.

\item F* är en vidareutveckling av F\# och används för att verifiera och bevisa
egenskaper hos program. En av de större skillnaderna från F\# är att F* har
stöd för beroendetyper. F* är en del av .Net vilket gör att bevis och kod som
är skriven i F* går att använda i all andra .Net språk. Stora delar av F* är
formaliserade och bevisade i Coq.
\end{itemize}

\begin{comment}
Källor och annat material
HOL http://www.cl.cam.ac.uk/~jrh13/hol-light/
Z3 http://research.microsoft.com/en-us/um/redmond/projects/z3/old/
F* http://research.microsoft.com/en-us/projects/fstar/
\end{comment}
