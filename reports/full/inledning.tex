\section{Inledning}
Komplexiteten hos matematiska bevis har med tiden ökat markant och bevis med
penna och papper är ofta inte realiserbart. Datorer kan i dessa fall utnyttjas
som hjälpmedel för att verifiera varje logiskt steg i bevisen. Flera satser har
formaliserat och två exempel är Fyrfärgssatsen\cite{gonthier2008formal} och
Feit-Thompsons sats\cite{something}. För att bevisen ska vara korrekta har man
formaliserat all tillhörande matematik.

Vidare gör bevisens komplexitet även att enbart områdesexperter klarar av att
granska och verifiera bevisen. Fermats stora sats är ett exempel på detta
problemet då man rapporterade beviset som korrekt 1993 men ett fel hittades
sedan 2 år senare\cite{something}.

Vårt projekt har haft som mål att studera datorassisterad formalisering och
sedan formalisera en matematisk algoritm för att få en inblick i detta
forskningsområdet.

\subsection{Formalisering med datorhjälp}
Formalisering är ett aktivt forskningsområde och det finns flertalet
datorverktyg för formalisering och verifiering av formella bevis som är under
aktiv utveckling.

Vi valde verktyget Coq. Coq är ett funktionellt programmeringsspråk inte helt
olikt Haskell. Men till skillnad från Haskell är Coq även en interaktiv
bevisassistent. Utöver detta ska vi även använda biblioteket SSReflect som är
en utökning av Coq.

\subsection{Multiplikation av polynom}
För att få tillämpning

\subsection{Mål}
Målet med vårat projekt är således att lära oss hur man använder Coq och dess
tillägg SSReflect. Vidare skall vi också använda dessa verktyg för att
formalisera ett bevis av Toom-Cook algoritmen.
