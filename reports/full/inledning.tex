\section{Inledning}
Formalisering av matematik

\subsection{Verktyg för formalisering av matematik}
Det finns många olika program och språk för att formallisera och bevisa
olika former av matematik eller logik. En sak som de flesta har gemensamt är
att de är uppbyggd på funktionella programmerings paradigmer.

\subsubsection{Agda}
Agda är utveckladat på chalmers och påminner till stor del om Haskell. Till
skillnad mot Coq så finns det inga inbyggda taktiker. Istället för att skriva
bevis med taktiker så skrivs bevisen på samma sätt som vanliga funktioner.
Detta gör att det är lättare att se att program och bevis är samma sak. Agda är
inte lika matematiskt inriktad som Coq utan används mer till att bevisa
korrekthet hos program. En fördel med Agda är att alla Unicode tecken är
tillåtna vilket gör det enkelt att skriva sina program och bevis på samma sätt
som man skulle göra det på paper.

\subsubsection{Z3}
Z3 är ett språk som har utvecklats av Microsoft för att förenkla och bevisa
olika theorem. Kan användas tillsammans med flera stora ickefunktionella språk
som Python,C och .NET.

\begin{itemize}
  \item Utvecklat av MIcrosoft
  \item ''Theorem prover''
  \item Används mest för att testa och verifiera program
\end{itemize}

\subsubsection{HOL-light}
HOL är en av de första teorem bevisarna och HOL-light som är en utveckling av
det används idag av Intel för att bevisa att vissa hårdvarukomponenter fungerar
korrekt.

\subsection{Tillämpningar av Coq}
De tre största tillämpningarna av Coq är formaliseringen av Fyrfärgssatsen och
Feit-Thompsonsatsen samt en C-kompilator som är skriven och bevisad i Coq.

\subsubsection{Fyrfärgssatsen}
Fyrfärgssatsen säger att, givet varje möjlig uppdelning av ett plan i
sammanhängande regioner, så krävs det högst fyra färger för att färglägga alla
regionerna så att inga angränsande regioner har samma färg. Två regioner anses
vara angränsande om de delar en gemensam kant som inte är ett hörn.

Satsen lades fram för första gången 1852 men visades först 1976 av Kenneth
Appel och Wolfgang Haken. De använde sig av en dator till hjälp när de bevisade
satsen vilket gjorde att beviset inte accepterades bland alla matematiker
eftersom det inte direkt enkelt kunde kontrolleras av en människa. År 1995
konstruerade Neil Robertson, Daniel Sanders, Paul Seymor och Robin Thomas ett
enklare bevis som var baserat på samma idé vilket blåste bort en del tvivel.
Detta bevis använde dock fortfarande datorer till hjälp och även om det gick
att kontrollera de delarna av beviset skrivet i text så kvarstod problemet med
att verifiera att koden var korrekt. En lösning till det här problemet är att
skriva ett formellt bevis för att bevisa att programmet är korrekt.

År 2000 försökte en forskningsgrupp ledd av Georges Gonthier att formellt
bevisa delar av koden i det förenklade beviset från 1995. De lyckades med
detta, men var sedan intresserade av att undersöka om hela beviset var korrekt,
vilket skulle innebära att de var tvungna att formalisera hela beviset, inte
bara koden. År 2005 var de klara med att formalisera hela beviset för
fyrfärgssatsen.

Referens: (Gonthier, Georges (2008), "Formal Proof—The Four-Color Theorem",
Notices of the American Mathematical Society 55 (11): 1382–1393)

\subsubsection{CompCerts C-kompilator}
CompCert\cite{compcert} är ett projekt som utforskar möjligheten att utveckla
formellt bevisade kompilatorer. Att kompilatorn är formellt bevisad innebär att
det finns ett matematiskt bevis, som kan kontrolleras genom en mekanisk check,
för att den exekverbara koden beter sig så som står föreskrivet i källkoden.
Rent konkret innebär detta att man är garanterad att den exekverbara koden inte
innehåller buggar som är skapade av kompilatorn.

Huvudresultatet av detta projekt är en fungerande C-kompilator som är skriven
och bevisad i Coq. Kompilatorn stödjer hela ANSI C med följande undantag:
\begin{itemize}
  \item Ostrukturerade switch-satser (t. ex Duffs maskin) är inte tillåtna.
  \item Alla funktioner måste ha en funktionsprototyp.
  \item Det går inte att definiera funktioner som kan ta ett
    varierande antal argument.
  \item Det finns ingen garanti för att longjmp och setjmp fungerar.
\end{itemize}
Vad beträffar prestanda så är den exekverbara koden genererad av
CompCerts C-kompilator något snabbare än den genererad av
C-kompilatorn i GCC (GNU Compiler Collection) utan optimeringar
(\texttt{-O0}) och något långsammare än den genererad av
C-kompilatorn i GCC med optimeringsnivå 1 (\texttt{-O1}). CompCerts
C-kompilators huvudsakliga syfte är dock inte att konkurrera mot
andra kompilatorer vad gäller prestanda utan utmärker sig genom
att vara fri från kompileringsfel.

\subsubsection{Feit-Thompsons sats}
Feit-Thompsons är en sats inom matematisk gruppteori som säger att en ändlig
grupp alltid är lösbar om dess ordning är udda. Denna sats bevisades av Walter
Feit och John Griggs Thompson 1963.

Beviset för Feit-Thompsons sats är stort och sträcker sig över två volymer, det
är mycket material för en person att sätta sig in i och verifiera för hand.
Storleken på beviset och därmed möjligheten till någon dold miss i beviset är
en av anledningarna till varför beviset för Feit-Thompsons sats är intressant
att formalisera i Coq. Feit-Thompsons sats är också intressant på det sättet
att det är en milstolpe på vägen att formalisera beviset för klassificeringen
av ändliga simpla grupper. Klassificeringen av ändliga simpla grupper är en
sats vars bevis består av tiotusentals sidor av flera hundra artiklar.

Efter att Georges Gonthier och hans forskningsgrupp var klara med det formella
beviset för fyrfärgssatsen så började de konstruera ett formellt bevis i Coq
för Feit-Thompsons sats. Detta var ett stort projekt som tog sex år och de var
färdiga med att formalisera hela beviset 2012.

I samband med formaliseringen av beviset så har även en stor del av matematisk
gruppteori verifierats.

Referens:
\begin{itemize}
  \item http://research.microsoft.com/en-us/news/features/gonthierproof-101112.aspx
  \item Aschbacher, Michael (2004). "The Status of the Classification of the
    Finite Simple Groups". Notices of the American Mathematical Society.
\end{itemize}

\subsection{SSReflect}
SSReflect är ett tillägg till Coq som utvecklades i samband med formaliseringen
av beviset för fyrfärgssatsen.

\begin{itemize}
  \item Small Scale Reflection
  \item Tre typer av steg
\end{itemize}
