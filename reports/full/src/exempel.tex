\section{Exempel på Toom-3}
Säg att vi vill multiplicera två polynom $p, q \in \mathbb{Z}_5[x]$\footnote{$\mathbb{Z}_5$ betecknar heltalen modulo 5.}.
$p, q \in \mathbb{Z}_5[x]$\footnote{$\mathbb{Z}_5$ betecknar heltalen modulo 5.}.
Vi låter 0, 1, 2, och 4 beteckna elementen i $\mathbb{Z}_5[x]$ och vi låter
\begin{align*}
  p(x) &= 4+x+3x^2+3x^3+4x^4 \\
  q(x) &= 1+2x+2x^2+x^3+3x^4
\end{align*}

\paragraph{Uppdelning}
Vårt b i uppdelningssteget som säger i hur stora delar vi ska dela upp
polynomen ges av:
\begin{equation*}
  b = \floor[\bigg]{\frac{1 + \mbox{grad p}}{\mbox{toom-instans}}} + 1 = \floor[\bigg]{\frac{1 + 4}{3}} + 1 = 2
\end{equation*}
Alltså delar vi polynomen vid potenser av $x$ som är multiplar av 2 och låter
delarna vara koefficienter i polynomen $u$ och $v$:
\begin{align*}
  u(y) &= (4+x)+(3 + 3x)y+ 4y^2 \\
  v(y) &= (1 + 2x)+ (2 + x)y + 3y^2
\end{align*}

\paragraph{Evaluering}
Om vi väljer evalueringspunkterna:
\begin{equation*}
  0, 1, 2, 3, 4
\end{equation*}
så får vi evalueringsmatrisen:
\begin{equation*}
  V_e = \begin{pmatrix}
    0^0 & 0^1 & 0^2 \\
    1^0 & 1^1 & 1^2 \\
    2^0 & 2^1 & 2^2 \\
    3^0 & 3^1 & 3^2 \\
    4^0 & 4^1 & 4^2
  \end{pmatrix} =
  \begin{pmatrix}
    1 & 0 & 0 \\
    1 & 1 & 1 \\
    1 & 2 & 4 \\
    1 & 3 & 4 \\
    1 & 4 & 1
  \end{pmatrix}
\end{equation*}
Genom att multiplicera evalueringsmatrisen med vektorerna av koefficienter för
våra polynom $u$ och $v$ får vi vektorer med polynomens värden i
evalueringspunkterna.
\begin{equation*}
  \begin{pmatrix}
    u(0) \\
    u(1) \\
    u(2) \\
    u(3) \\
    u(4)
  \end{pmatrix} =
  \begin{pmatrix}
    1 & 0 & 0 \\
    1 & 1 & 1 \\
    1 & 2 & 4 \\
    1 & 3 & 4 \\
    1 & 4 & 1
  \end{pmatrix}
  \begin{pmatrix}
    4 + x \\
    3 + 3x \\
    4
  \end{pmatrix} =
  \begin{pmatrix}
    4 + x \\
    1 + 4x \\
    1 + 2x \\
    4 \\
    3x
  \end{pmatrix}
\end{equation*}
\begin{equation*}
  \begin{pmatrix}
    v(0) \\
    v(1) \\
    v(2) \\
    v(3) \\
    v(4)
  \end{pmatrix} =
  \begin{pmatrix}
    1 & 0 & 0 \\
    1 & 1 & 1 \\
    1 & 2 & 4 \\
    1 & 3 & 4 \\
    1 & 4 & 1
  \end{pmatrix}
  \begin{pmatrix}
    1 + 2x \\
    2 + x \\
    3
  \end{pmatrix} =
  \begin{pmatrix}
    1 + 2x \\
    1 + 3x \\
    2 + 4x \\
    4 \\
    2 + x
  \end{pmatrix}
\end{equation*}

\paragraph{Rekursiv multiplikation}
Genom att multiplicera $u$ och $v$ i de evaluerade punkterna rekursivt får vi
värdet i evalueringspunkterna av $u \cdot v = w$, det polynom som skall
interpoleras fram:
\begin{equation*}
  \begin{pmatrix}
    w(0) \\
    w(1) \\
    w(2) \\
    w(3) \\
    w(4)
  \end{pmatrix} =
  \begin{pmatrix}
    u(0) \cdot v(0) \\
    u(1) \cdot v(1) \\
    u(2) \cdot v(2) \\
    u(3) \cdot v(3) \\
    u(4) \cdot v(4)
  \end{pmatrix} =
  \begin{pmatrix}
    4 + 4x + 2x^2 \\
    1 + 2x + 2x^2 \\
    2 + 3x + 3x^2 \\
    1 \\
    x + 3x^2
  \end{pmatrix}
\end{equation*}

\paragraph{Interpolation}
Vi vet att med:
\begin{equation*}
  V_I = \begin{pmatrix}
    0^0 & 0^1 & 0^2 & 0^3 & 0^4 \\
    1^0 & 1^1 & 1^2 & 1^3 & 1^4 \\
    2^0 & 2^1 & 2^2 & 2^3 & 2^4 \\
    3^0 & 3^1 & 3^2 & 3^3 & 3^4 \\
    4^0 & 4^1 & 4^2 & 4^3 & 4^4
  \end{pmatrix} =
  \begin{pmatrix}
    1 & 0 & 0 & 0 & 0 \\
    1 & 1 & 1 & 1 & 1 \\
    1 & 2 & 4 & 3 & 1 \\
    1 & 3 & 4 & 2 & 1 \\
    1 & 4 & 1 & 4 & 1
  \end{pmatrix}
\end{equation*}
Så är:
\begin{equation*}
  V_I \cdot \begin{pmatrix}
    w_0 \\
    w_1 \\
    w_2 \\
    w_3 \\
    w_4
  \end{pmatrix} =
  \begin{pmatrix}
    w(0) \\
    w(1) \\
    w(2) \\
    w(3) \\
    w(4)
  \end{pmatrix}
\end{equation*}
Där $w_0, ..., w_4$ är koefficienterna i polynomet $w$. Enligt formeln för en
Vandermondematris determinant så är
\begin{align*}
 \det V_I &= (0-1)(0-2)(0-3)(0-4)(1-2)(1-3)(1-4)(2-3)(2-4)(3-4) \\
          &= 3  \\
          &\neq 0,
\end{align*}
så $V_I$ är inverterbar eftersom varje nollskiljt element i $\mathbb{Z}_5[x]$ har en multiplikativ invers.
Alltså kan vi få
koefficienterna i $w$ genom:
\begin{align*}
  \begin{pmatrix}
    w_0 \\
    w_1 \\
    w_2 \\
    w_3 \\
    w_4
  \end{pmatrix} & =
  V_I^{-1} \cdot \begin{pmatrix}
    w(0) \\
    w(1) \\
    w(2) \\
    w(3) \\
    w(4)
  \end{pmatrix} =
  \begin{pmatrix}
    1 & 0 & 0 & 0 & 0 \\
    0 & 4 & 2 & 3 & 1 \\
    0 & 4 & 1 & 1 & 4 \\
    0 & 4 & 3 & 2 & 1 \\
    4 & 4 & 4 & 4 & 4
  \end{pmatrix}
  \begin{pmatrix}
    w(0) \\
    w(1) \\
    w(2) \\
    w(3) \\
    w(4)
  \end{pmatrix} = \\
  &= \begin{pmatrix}
    4 + 4x + 2x^2 \\
    1 + 2x^2 \\
    2 + 3x^2 \\
    2 + 3x \\
    2
  \end{pmatrix}
\end{align*}
Vi har då att $w$ är:
\begin{equation*}
  w(y) = (4 + 4x + 2x^2) + (1 + 2x^2)y + (2 + 3x^2)y^2 + (2 + 3x)y^3 + 2y^4
\end{equation*}

\paragraph{Sammansättning}
Produkten $p(x) \cdot q(x)$ fås slutligen genom att beräkna värdet av $w$ i
$x^b$, alltså i $x^2$ eftersom $b = 2$ i det här fallet:
\begin{align*}
  w(x^2) =& (4 + 4x + 2x^2) + (1 + 2x^2)x^2 + (2 + 3x^2)(x^2)^2 + (2 + 3x)(x^2)^3 + 2(x^2)^4 \\
         =& 4 + 4x + 2x^2 + x^2 + 2x^4 + 2x^4 + 3x^6 + 2x^6 + 3x^7 + 2x^8 \\
         =& 4 + 4x + 3x^2 + 4x^4 + 3x^7 + 2x^8,
\end{align*}
vilket är samma resultat som vi hade fått vid naiv polynommultiplikation.
