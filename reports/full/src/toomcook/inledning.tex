Toom-Cook är en algoritm för att multiplicera två polynom och är namngiven efter
Andrei Toom och Stephen Cook.

Algoritmen är intressant eftersom den har en bättre asymptotisk tidskomplexitet
än naiv\footnote{Enligt definitionen i appendix~\ref{sec:algebra}}
polynommultiplikation där man multiplicerar varje term i det ena polynomet med
varje term i det andra polynomet vilket har
tidskomplexiteten\footnote{Funktionen $T(n)$ är $\Ordo(f(n))$ om det existerar
konstanter $c > 0$ och $n_0 \geq 0$ så att $T(n) \leq c \cdot f(n)$ för alla $n
\geq n_0$. $\Ordo(\cdot )$-notationen beskriver funktionens asymptotiska
beteende, det vill säga hur funktionen växer med ökande storlek på argumentet.}
$\Ordo\left(n^2\right)$, där $n$ är graden på det största av de två polynomen
som skall multipliceras. Eftersom problemet att multiplicera två heltal kan
reduceras till att multiplicera två polynom kan algoritmen även användas för
heltalsmultiplikation. Denna reduktion kan göras utan att den asymptotiska
tidskomplexiteten försämras. Detta innebär att man kan uppnå en bättre
asymptotisk tidskomplexitet än den för naiv heltalsmultiplikation som lärs ut i
grundskolan och har tidskomplexiteten $\Ordo\left(n^2\right)$, där $n$ är
antalet siffror i det största talet.

Det finns flera varianter av Toom-Cook. \toomp{m} är en enskild instans av
Toom-Cook som delar polynomen som skall multipliceras i $m$ delar. Vanligtvis
när man talar om Toom-Cook syftar man på \toomp{3}. Ett intressant specialfall
av Toom-Cook är \toomp{2} som under vissa förutsättningar svarar mot
Karatsuba-algoritmen vilken beskrivs i kapitel~\ref{sec:karatsuba}.

\toomp{m} har tidskomplexiteten $\Ordo \left(n^{\log_2(2 m-1)/\log_2 m}\right)$
\cite{bodrato2007towards}, där $m$ är graden på det största polynomet.
Konstanten som döljs av $\Ordo(\cdot)$-notationen växer med $m$ och har en
betydande praktisk inverkan. För heltalsmultiplikation finns algoritmer som
bygger på diskret fouriertransform och har en ännu bättre asymptotisk
tidskomplexitet, till exempel Schönhage-Strassen-algoritmen.

Både Toom-Cook och algoritmer som bygger på diskret fouriertransform används i
praktiken. I till exempel GMP, The GNU Multiple Precision Arithmetic Library,
används Schönhage-Strassen-algoritmen samt olika instanser av
Toom-Cook-algoritmen för multiplikation av heltal \cite{gmpdoc}.
