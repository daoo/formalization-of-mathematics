Toom-Cook är en algoritm för att multiplicera två polynom och är namngiven
efter Andrei Toom och Stephen Cook.

Algoritmen är intressant eftersom den har en bättre asymptotisk tidskomplexitet
än naiv polynommultiplikation där man multiplicerar varje term i det ena
polynomet med varje term i det andra polynomet vilket har den asymptotiska
tidskomplexiteten $O\left(n^2\right)$. Eftersom problemet att multiplicera två
heltal kan reduceras till att multiplicera två polynom kan algoritmen även
användas för heltalsmultiplikation, denna reduktion kan göras utan att den
asymptotiska tidskomplexiteten försämras. Detta innebär att man kan uppnå en
bättre asymptotisk tidskomplexitet än den för naiv heltalsmultiplikation som
lärs ut i grundskolan och har den asymptotiska tidskomplexiteten
$O\left(n^2\right)$.

Toom-Cook är en $"$divide and conquer$"$-algoritm och bygger på att man delar
upp polynomen som skall multipliceras i mindre delar, dessa delar får sedan stå
som koefficienter i två nya polynom som evalueras i olika punkter för att sedan
punktvis multipliceras och därefter interpoleras tillbaka till ett nytt
polynom. Genom att evaluera det nya polynomet i en punkt som svarar mot hur
stora delar de ursprungliga polynomen delades upp i så får man slutligen
produkten. Detta är dock bara de ingående stegen, för en mer detaljerad
beskrivning av algoritmen se följande undersektion.

Det finns flera varianter
av Toom-Cook. Toom-k är en enskild instans av Toom-Cook som delar polynomen som
skall multipliceras i k delar. Vanligtvis när man talar om Toom-Cook syftar man
på Toom-3. Det finns även Toom-versioner som delar upp polynomen i olika antal
delar. Två intressanta specialfall av Toom-Cook är Toom-1 som svarar mot naiv
polynommultiplikation och Toom-2, Karatsuba-algoritmen. Toom-k har den
asymptotiska tidskomplexiteten $O(n^{log_2(2 k-1)/log_2 k})$, men konstanten
som döljs av ordo-notationen växer med k och har en betydande praktisk
inverkan. För heltalsmultiplikation finns algoritmer som bygger på diskret
fouriertransform och har en ännu bättre asymptotisk tidskomplexitet, t ex
Schönhage-Strassen-algoritmen.

Både Toom-Cook och algoritmer som bygger på diskret fouriertransform används i
praktiken. I t ex gmplib, The GNU Multiple Precision Arithmetic Library,
används Schönhage-Strassen-algoritmen samt olika instanser av
Toom-Cook-algoritmen för multiplikation av heltal.
