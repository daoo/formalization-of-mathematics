$\toom$ är en algoritm för att multiplicera två polynom och är namngiven efter
Andrei Toom och Stephen Cook.

Algoritmen är intressant eftersom den har en bättre asymptotisk tidskomplexitet
än naiv polynommultiplikation där man multiplicerar varje term i det ena
polynomet med varje term i det andra polynomet vilket har den asymptotiska
tidskomplexiteten $O\left(n^2\right)$ \footnote{Funktionen $T(n)$ är $O(f(n))$
om det existerar konstanter $c > 0$ och $n_0 \geq 0$ så att $T(n) \leq c \cdot
f(n)$ för alla $n \geq n_0$.}, där n är graden på det största av de två
polynomen som skall multipliceras. Eftersom problemet att multiplicera två
heltal kan reduceras till att multiplicera två polynom kan algoritmen även
användas för heltalsmultiplikation, denna reduktion kan göras utan att den
asymptotiska tidskomplexiteten försämras. Detta innebär att man kan uppnå en
bättre asymptotisk tidskomplexitet än den för naiv heltalsmultiplikation som
lärs ut i grundskolan och har den asymptotiska tidskomplexiteten
$O\left(n^2\right)$, där n är antalet siffror i det största talet.

Det finns flera varianter av $\toom$. $\toom - k$ är en enskild instans av
$\toom$ som delar polynomen som skall multipliceras i $k$ delar. Vanligtvis när
man talar om $\toom$ syftar man på Toom-3. Det finns även Toom-versioner som
delar upp polynomen i olika antal delar. Ett intressant specialfall av $\toom$
är Toom-2 som under vissa förutsättningar svarar mot Karatsuba-algoritmen.

Toom-$k$ har den asymptotiska tidskomplexiteten $O(n^{log_2(2 k-1)/log_2 k})$,
men konstanten som döljs av ordo-notationen växer med $k$ och har en betydande
praktisk inverkan. För heltalsmultiplikation finns algoritmer som bygger på
diskret fouriertransform och har en ännu bättre asymptotisk tidskomplexitet, t
ex Schönhage-Strassen-algoritmen.

Både $\toom$ och algoritmer som bygger på diskret fouriertransform används i
praktiken. I t ex gmplib, The GNU Multiple Precision Arithmetic Library,
används Schönhage-Strassen-algoritmen samt olika instanser av
$\toom$-algoritmen för multiplikation av heltal.
