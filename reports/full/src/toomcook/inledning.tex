$\toom$ är en algoritm för att multiplicera två polynom och är namngiven efter
Andrei Toom och Stephen Cook.

Algoritmen är intressant eftersom den har en bättre asymptotisk tidskomplexitet
än naiv polynommultiplikation där man multiplicerar varje term i det ena
polynomet med varje term i det andra polynomet vilket har tidskomplexiteten
$O\left(n^2\right)$ \footnote{Funktionen $T(n)$ är $O(f(n))$ om det existerar
konstanter $c > 0$ och $n_0 \geq 0$ så att $T(n) \leq c \cdot f(n)$ för alla $n
\geq n_0$. $O(\cdot )$-notationen beskriver funktionens asymptotiska beteende,
det vill säga hur funktionen växer med ökande storlek på argumentet.}, där $n$
är graden på det största av de två polynomen som skall multipliceras. Eftersom
problemet att multiplicera två heltal kan reduceras till att multiplicera två
polynom kan algoritmen även användas för heltalsmultiplikation, denna reduktion
kan göras utan att den asymptotiska tidskomplexiteten försämras. Detta innebär
att man kan uppnå en bättre asymptotisk tidskomplexitet än den för naiv
heltalsmultiplikation som lärs ut i grundskolan och har tidskomplexiteten
$O\left(n^2\right)$, där $n$ är antalet siffror i det största talet.

Det finns flera varianter av $\toom$. $\toomp{k}$ är en enskild instans av
$\toom$ som delar polynomen som skall multipliceras i $k$ delar. Vanligtvis när
man talar om $\toom$ syftar man på $\toomp{3}$. Det finns även versioner av
$\toom$ som delar upp polynomen i olika antal delar. Ett intressant specialfall
av $\toom$ är $\toomp{2}$ som under vissa förutsättningar svarar mot
Karatsuba-algoritmen.

$\toomp{k}$ har tidskomplexiteten $O(n^{log_2(2 k-1)/log_2 k})$
\cite{bodrato2007towards}, men konstanten som döljs av ordo-notationen växer
med $k$ och har en betydande praktisk inverkan. För heltalsmultiplikation
finns algoritmer som bygger på diskret fouriertransform och har en ännu bättre
asymptotisk tidskomplexitet, till exempel Schönhage-Strassen-algoritmen.

Både $\toom$ och algoritmer som bygger på diskret fouriertransform används i
praktiken. I till exempel gmplib, The GNU Multiple Precision Arithmetic
Library, används Schönhage-Strassen-algoritmen samt olika instanser av
$\toom$-algoritmen för multiplikation av heltal \cite{gmpdoc}.
