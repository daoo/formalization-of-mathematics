\section{Definition av Toom-Cook}
\label{sec:definition}
Det första steget i formaliseringen av Toom-Cook-algoritmen är en informell men
detaljerad definition och bevis av algoritmen och att den är korrekt. Detta
avsnitt är ägnat åt den. Versionen av algoritmen som presenteras bygger på den
i \cite{bodrato2007notes} och presentationen av den följer källans. För en
introduktion till integritetsområden och polynom, se
appendix~\ref{app:algebra}.

Låt R vara ett integritetsområde och låt $p, q \in R[x]$, där
\begin{align*}
  p(x) &= a_0 + a_1 x + \cdots + a_n x^n, \\
  q(x) &= b_0 + b_1 x + \cdots + b_s x^s
\end{align*}
med $0 \leq s \leq n$.
I detta avsnitt definierar vi $\toomm (p, q)$, för naturliga tal $m \geq 3$,
som resultatet av algoritmen nedan.

\subsection{Gradkontroll}
\label{sec:gradkontroll}
Om $\grad p = n \leq 2$, låt $\toomm (p, q) = p \cdot q$, där $p\cdot q$
beräknas med naiv polynommultiplikation, annars gå vidare till steg
\ref{sec:uppdelning}.

\subsection{Uppdelning}
\label{sec:uppdelning}
Låt
\begin{align}
  b &= \floor[\bigg]{\frac{1 + \grad p}{m}} + 1 = \floor[\bigg]{\frac{1 + n}{m}} + 1. \label{eq:b}
\intertext{För $f \in R[x]$ och}
  f &= q x^k + r
\intertext{med $q, r \in R[x]$ och $r = 0$ eller $\grad r \leq \grad x^k$, låt}
  f/x^k       &= q \label{eq:rdiv} \\
  f \modu x^k &= r. \label{eq:rmod}
\intertext{Nu definierar vi $u, v \in \left(R[x]\right)[y]$. Låt}
  u(y) &= u_0 + u_1 y + \cdots + u_{m-1} y^{m-1} \label{eq:u}
\intertext{där}
  u_k &= \left(p / x^{bk} \right) \modu x^b
\intertext{och}
  v(y)&=v_0 + v_1 y + \cdots + v_{m-1} y^{m-1} \label{eq:v}
\intertext{där}
  v_k &= \left(q / x^{bk} \right) \modu x^b
\end{align}
Vi vill alltså att $u(x^b)=p(x)$ och $v(x^b)=q(x)$.

\subsection{Evaluering}
\label{sec:evaluering}
Nu ska vi beräkna $w = u \cdot v$. Vi kommer göra detta genom interpolation i
steg~\ref{sec:interpol}. Eftersom
\begin{align*}
 \grad w = \grad u + \grad v \leq m - 1 + m - 1 = 2m-2 = d
\end{align*}
kan koefficienterna i $w$ bestämmas om vi vet värdet av $w$ i $d + 1$ skiljda
punkter. Vi evaluerar $w$ i punkterna $\alpha_0, ...,  \alpha_d$, där
$\alpha_i \in R$. För att göra det beräknar vi i detta steg först $u(\alpha_i)$
och $v(\alpha_i)$ genom matrismultiplikation. I nästa steg beräknas sedan
$w(\alpha_i)=u(\alpha_i) \cdot v(\alpha_i)$. Låt

\begin{align*}
 V_e &=
\begin{pmatrix}
  \alpha_0^0 & \alpha_0^1 & \cdots & \alpha_0^{m-1} \\
  \vdots     & \vdots     & \ddots & \vdots         \\
  \alpha_d^0 & \alpha_d^1 & \cdots & \alpha_d^{m-1}
\end{pmatrix}.
\intertext{Då får vi}
  V_e \cdot
  \begin{pmatrix}
    u_0    \\
    \vdots \\
    u_{m-1}
  \end{pmatrix}
  &=
  \begin{pmatrix}
    u(\alpha_0) \\
    \vdots      \\
    u(\alpha_d)
  \end{pmatrix}
\end{align*}
och motsvarande för $v$.

\subsection{Rekursiv multiplikation}
\label{sec:rekursiv}
Vi beräknar $w(\alpha_i)=u(\alpha_i) \cdot v(\alpha_i)$ för i $= 0, \dots, d$
rekursivt genom att anropa Toom-Cook med $u(\alpha_i)$ och $v(\alpha_i)$ som
argument. Notera att $u(\alpha_i)$, $v(\alpha_i) \in R[x]$.

\subsection{Interpolation}
\label{sec:interpol}
Vi bestämmer koefficienterna i $w(y)=w_0 + w_1 y + \ldots + w_d y^d$ genom
interpolation. Vi vet värdet av $w(y) = u(y) \cdot v(y)$ i $d + 1$ punkter. Om

\begin{align}
  \label{eq:NAME3}
  V_I &=
  \begin{pmatrix}
    \alpha_0^0 & \alpha_0^1 & \cdots & \alpha_0^d \\
    \vdots     & \vdots     & \ddots & \vdots     \\
    \alpha_d^0 & \alpha_d^1 & \cdots & \alpha_d^d
  \end{pmatrix}
\intertext{så är}
  \label{eq:NAME4}
  V_I \cdot
  \begin{pmatrix}
    w_0    \\
    \vdots \\
    w_d
  \end{pmatrix}
  &=
  \begin{pmatrix}
    w(\alpha_0) \\
    \vdots      \\
    w(\alpha_d)
  \end{pmatrix}
\intertext{och därmed är}
  \label{eq:NAME5}
  \begin{pmatrix}
    w_0    \\
    \vdots \\
    w_d
  \end{pmatrix} &=
  V_I^{-1} \cdot
  \begin{pmatrix}
    w(\alpha_0) \\
    \vdots      \\
    w(\alpha_d)
  \end{pmatrix}
\intertext{förutsatt att $V_I$ är inverterbar. Detta är den om determinanten
$V_I$ är ett invertarbart element i $R$ \ref{somestuff}. Eftersom $V_I$ är en
Vandermondematris så är}
  \label{eq:NAME6}
  \det V_I &= \prod_{0 \leq i < j \leq d} (\alpha_i - \alpha_j)
\end{align}

\subsection{Sammansättning}
\label{sec:sammansattning}
Vi får slutligen $p(x) \cdot q(x)$ genom att evaluera $w$ i $x^b$ eftersom
$w(x^b) = u(x^b) \cdot v(x^b)$.
