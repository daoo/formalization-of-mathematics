\section{Definition av Toom-Cook-$m$}
I detta avsnitt definierar vi Toom-Cook m $(p, q)$, för $m \in \mathbb{N}$ och
$m \geq 3$, som resultatet av algoritmen nedan. Versionen av algoritmen som
presenteras nedan bygger på \cite{bodrato2007a} och presentationen av den
följer källans. För en introduktion till integritetsområden och polynom, se
APPENDIX NÅNTING.

Låt R vara ett integritetsområde och låt $p, q \in R[x]$, där
\begin{align*}
  &p(x) = a_0 + a_1 x + ... + a_n x^n, \\
  &q(x) = b_0 + b_1 x + ... + b_s x^s
\end{align*}
med $0 \leq s \leq n$.

\subsection{Gradkontroll}
Om $\grad p = n \leq 2$, låt Toom-Cook $m (p, q) = p \cdot q$, annars gå vidare
till steg 2.

\subsection{Uppdelning}
\label{uppdelning}
Låt
\begin{align*}
  b &= \floor[\bigg]{\frac{1 + \grad p}{m}} + 1 = \floor[\bigg]{\frac{1 + n}{m}} + 1.
\intertext{För $f \in R[x]$ och}
  f &= q x^k + r
\intertext{med $q, r \in R[x]$ och $r = 0$ eller $\grad r \leq \grad x^k$, låt}
  f/x^k &= q \\
  f \modu x^k &= r.
\intertext{Nu definierar vi $u, v \in R[x][y]$. Låt}
  u(y)&=u_0 + u_1 y + ... + u_{m-1} y^{m-1}
\intertext{där}
  u_k &= p / x^{bk} \modu x^b
\intertext{och}
  v(y)&=v_0 + v_1 y + ... + v_{m-1} y^{m-1}
\intertext{där}
  v_k &= q / x^{bk} \modu x^b
\end{align*}
Vi vill att $u(x^b)=p(x)$ och $v(x^b)=q(x)$.

\subsection{Evaluering}
Nu ska vi beräkna $w = u \cdot v$. Vi gör detta genom interpolation. Vi
beräknar $w(\alpha_i)=u(\alpha_i) \cdot v(\alpha_i)$ för $d + 1$ punkter
$\alpha_0, ...,  \alpha_d$, där $\alpha_i \in R$ och

\begin{align*}
  d = m - 1 + m -1 = 2m-2 \geq \grad w = \grad u + \grad v.
\end{align*}
Därefter bestäms koefficienterna i $w$ genom interpolation. I detta steg
beräknar vi $u(\alpha_i)$ och $v(\alpha_i)$. Låt
\begin{align*}
  V_e &=
  \begin{pmatrix}
    \alpha_0^0 & \alpha_0^1 & ... & \alpha_0^{m-1} \\
    \vdots     & \vdots     &     & \vdots         \\
    \alpha_d^0 & \alpha_d^1 & ... & \alpha_d^{m-1}
  \end{pmatrix}.
\intertext{Då får vi}
  V_e \cdot
  \begin{pmatrix}
    u_0    \\
    \vdots \\
    u_{m-1}
  \end{pmatrix}
  &=
  \begin{pmatrix}
    u(\alpha_0) \\
    \vdots      \\
    u(\alpha_d)
  \end{pmatrix}
\end{align*}
och motsvarande för $v$.

\subsection{Rekursiv multiplikation}
Vi beräknar $w(\alpha_i)=u(\alpha_i) \cdot v(\alpha_i)$ för i $= 0, ... , d$
rekursivt genom att anropa algoritmen med $u(\alpha_i)$ och $v(\alpha_i)$ som
argument. Notera att $u(\alpha_i)$, $v(\alpha_i) \in R[x]$.

\subsection{Interpolation}
Vi bestämmer koefficienterna i $w(y)=w_0 + w_1 y + \ldots + w_d y^d$ genom
interpolation. Om

\begin{align}
  \label{eq:NAME3}
  V_I &=
  \begin{pmatrix}
    \alpha_0^0 & \alpha_0^1 & ... & \alpha_0^d \\
    \vdots     & \vdots     &     & \vdots     \\
    \alpha_d^0 & \alpha_d^1 & ... & \alpha_d^d
  \end{pmatrix}
\intertext{så är}
  \label{eq:NAME4}
  V_I \cdot
  \begin{pmatrix}
    w_0    \\
    \vdots \\
    w_d
  \end{pmatrix}
  &=
  \begin{pmatrix}
    w(\alpha_0) \\
    \vdots      \\
    w(\alpha_d)
  \end{pmatrix}
\intertext{och därmed är}
  \label{eq:NAME5}
  \begin{pmatrix}
    w_0    \\
    \vdots \\
    w_d
  \end{pmatrix} &=
  V_I^{-1} \cdot
  \begin{pmatrix}
    w(\alpha_0) \\
    \vdots      \\
    w(\alpha_d)
  \end{pmatrix}
\intertext{förutsatt att $V_I$ är inverterbar. Detta är den om $\det V_I$ är
ett invertarbart elemet i $R$ [referens]. Eftersom $V_I$ är en
Vandermondematris så är}
  \label{eq:NAME6}
  \det V_I &= \prod_{0 \leq i < j \leq d} (\alpha_i - \alpha_j)
\end{align}

\subsection{Sammansättning}
Vi får slutligen det önskade resultatet $p(x) \cdot q(x)$ genom att evaluera
$w$ i $x^b$.
