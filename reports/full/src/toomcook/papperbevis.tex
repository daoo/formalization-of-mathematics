\section{Informellt bevis av Toom-Cook}
\label{sec:informhuvudbevis}
I detta stycke visar vi att för $p, q \in R[x]$ så är
$p \cdot q = Toom-Cook (p, q)$. Vi gör detta med hjälp av två lemman, vars bevis
kommer efter beviset av~\ref{prop:1}.

\begin{proposition}
  \label{prop:1}
  Antag att $R$ är ett integritetsområde och att $p, q \in R[x]$. Om det finns
  $\alpha_0, \dots, \alpha_{2m-2} \in R$ så att $\prod_{0 \leq i < j \leq d}
  (\alpha_i - \alpha_j)$ är inverterbar i $R$, så är med dessa punkter som
  interpolationspunkter

  \begin{equation}
    \label{eq:name7}
    Toom-Cook (p, q) =  p \cdot q.
  \end{equation}
\end{proposition}

\begin{proof}
Vi visar propositionen med induktion över $\grad p$.

\paragraph{Basfall.}
När $n = \grad p \leq 2$ så gäller (\ref{eq:name7}) enligt steg
\ref{in:gradkontroll} i algoritmen eftersom att vi räknar ut produkt direkt.

\paragraph{Induktionssteg.}
\label{sec:induktion}
Antag att $p, q \in R[x]$, där
\begin{align*}
  p(x) &= a_0 + a_1 x + \cdots + a_n x^n, \\
  q(x) &= b_0 + b_1 x + \cdots + b_s x^s
\end{align*}
med $0 \leq s \leq n$.

\bigskip\noindent
Antag också att $n > 2$ och att $Toom-Cook (f, g) =  f \cdot g$ gäller för polynom
av grad mindre än $n$. Eftersom $n > 2$ så går vi vidare till steg
\ref{in:gradkontroll} i algoritmen och skapar $u$ och $v$ från $p$ och $q$. När
detta är gjort evaluerar vi i steg 3 $u$ och $v$ i punkterna $\alpha_0, \dots,
\alpha_{2m-2}$ genom att multiplicera vektorn av deras koefficienter med
evalueringsmatrisen. I steg~\ref{in:rekursiv} anropar vi Toom-Cook igen, nu med
argumenten $u(\alpha_i) \cdot v(\alpha_i)$ för $i = 0, \ldots , 2m-2$. Dessa är
polynom i $R[x]$. Eftersom $\grad u(\alpha_i)$, $\grad v(\alpha_i) < n$ enligt
lemma~\ref{lemma:1} så är $Toom-Cook (u(\alpha_i), v(\alpha_i)) = u(\alpha_i) \cdot
v(\alpha_i)$ enligt induktionsantagnandet. I steg~\ref{in:interpol} skall vi
bestämma koefficienterna i $w(y)=u(y) \cdot v(y)$. Detta gör vi genom att lösa
matrisekvationen (\ref{eq:NAME4}). Eftersom interpolationsmatrisen $V_I$ enligt
antagande är inverterbar så ges koefficienterna entydigt av (\ref{eq:NAME5}). I
steg~\ref{in:sammansattning} evaluerar vi $w(y)=u(y) \cdot v(y)$ i $x^b$. Lemma
\ref{lemma:2} ger att $u(x^b)=p(x)$ och att $v(x^b)=q(x)$. Då $w(y)=u(y) \cdot
v(y)$ så är $w(x^b)=u(x^b) \cdot v(x^b)=p(x) \cdot q(x)$.
\end{proof}

\begin{lemma}
  \label{lemma:1}
  Antag att $p(x) \in R[x]$ och att $\grad p \geq 2$. Antag också att $u(y)$ är
  definierad enligt steg~\ref{in:uppdelning}, uppdelning, i algoritmen och att
  $\alpha \in R$. Då är $\grad u(\alpha) < \grad p$.
\end{lemma}
\begin{proof}
  Eftersom $\alpha \in R$ så är
  \begin{align*}
    \grad u(\alpha) &= \grad (u_0 + u_1 \alpha + ... + u_{m-1}\alpha^{m-1}) = \max \grad u_k,
  \end{align*}
  där $k={0,1,...,m-1}$.

  \bigskip\noindent
  $u_k = p(x)/x^{k b} \modu x^b$, så enligt definition av mod så är $\grad u_k
  < b$. Nu återstår att visa att $b < \grad p = n$. Vi har definierat $b =
  \floor[\big]{\frac{1 + n}{m}} + 1$. Om $n \geq 2$ och $m \geq 3$ så är
  \begin{align*}
    n-b &= n - \left( \floor[\bigg]{\frac{1 + n}{m}} + 1 \right) \geq \frac{nm-n-m-1}{m} \geq 0,
  \end{align*}
  så $\grad u(\alpha) < b \leq \grad p(x) = n$.
\end{proof}

\begin{lemma}
  \label{lemma:2}
  Antag att $p(x) \in R[x]$ och att $b$ och $u(y)$ är definierade som i steg 2
  i algoritmen. Då är $u(x^b)=p(x)$.
\end{lemma}
\begin{proof}
  Vi har att
  \begin{align*}
    u(x^b) &= \sum_{i = 0}^{m-1} p(x)/x^{bi} (\modu x^b) x^{bi} \\
           &= \sum_{i = 0}^{m-2} p(x)/x^{bi} (\modu x^b) x^{bi} + p(x)/x^{b(m-1)} (\modu x^b) x^{b (m-1)}.
  \end{align*}
  \paragraph{Påstående 1.} Den sista termen i $u(x^b)$ kan skrivas om till
  $p(x)/x^{b (m-1)} x^{b (m-1)}$ eftersom $\grad p - b (m-1) < b$.

  Med $\grad p = n$ och $b = \floor[\big]{\frac{1 + n}{m}} + 1$ har vi att
  \begin{align*}
    b - (n - b(m-1)) &= m b - n \\
                     &= m \left( \floor[\bigg]{\frac{1 + n}{m}} + 1 \right) - n \\
                     &= m \left( \frac{1 + n}{m} - \left\{ \frac{1 + n}{m} \right\} \right) + m - n \\
                     &= 1 + n - m \left\{ \frac{1 + n}{m} \right\} + m - n \\
                     &= 1 + m \left( 1 - \left\{ \frac{1 + n}{m} \right\} \right) > 0
  \end{align*}
  eftersom $0 < 1 - \{ x \} \leq 1$ för alla reella tal $x$
  \footnote{$\{x\}$ betecknar $x-\lfloor x \rfloor$. Se \cite{graham1989concrete}.}.
  Graden av $p(x)/x^{b(m-1)}$ är $\max \{n - b(m-1),0\}$ som är mindre än $b$,
  och därmed är $p(x)/x^{b(m-1)}  \modu x^b = p(x)/x^{b(m-1)}$, vilket visar
  Påstående 1. Så
  \begin{align*}
    u(x^b) &= \sum_{i = 0}^{m-2} p(x)/x^{bi} (\modu x^b) x^{bi} + p(x)/x^{b(m-1)} x^{b(m-1)}.
  \end{align*}
  För alla polynom $a(x)$ gäller att
  \begin{align}
    a(x) = a(x) \modu x^b + \left(a(x)/x^b\right) x^b \label{eq:name8}
  \end{align}
  då $a(x) \modu x^b$ och $p(x)/x^{b}$ är resten respektive kvoten vid division
  med $x^b$. Så om vi kan visa att $u(x^b) = p(x) \modu x^b + p(x)/x^{b} x^{b}$
  så är $u(x^b) = p(x)$. Och om sedan
  \begin{align}
    \sum_{i = 0}^{k} p(x)/x^{bi}  (\modu x^b) x^{bi} + (p(x)/x^{b(k + 1)})  x^{b(k + 1)} = \label{eq:name9} \\
    \sum_{i = 0}^{k-1} p(x)/x^{bi}  (\modu x^b) x^{bi} + (p(x)/x^{b k})  x^{b k} \label{eq:name10}
  \end{align}
  för $k \geq 1$ så ger induktion över $k$ att $u(x^b) =  p(x)$. Nu återstår
  alltså bara att visa att (\ref{eq:name9}) = (\ref{eq:name10}).

  \begin{align*}
    \sum_{i = 0}^{k} p(x)/x^{bi} (\modu x^b) x^{bi} &+ p(x)/x^{b(k + 1)}  x^{b(k + 1)} =\\
    \sum_{i = 0}^{k-1} p(x)/x^{bi} (\modu x^b) x^{bi} &+
    \left(p(x)/x^{b k} \modu x^b\right) x^{b k} + \left(p(x)/x^{b(k + 1)}\right)  x^{b(k + 1)} = \\
    \sum_{i = 0}^{k-1} p(x)/x^{bi} (\modu x^b) x^{bi} &+
    \left(p(x)/x^{b k} \modu x^b + \left(p(x)/x^{b(k + 1)}\right)  x^b \right) x^{b k} = \\
    \sum_{i = 0}^{k-1} p(x)/x^{bi} (\modu x^b) x^{bi} &+
    (p(x)/x^{b k} \modu x^b + ((p(x)/x^{b k})/x^b)  x^b) x^{b k} =
    \tag{på grund av (\ref{eq:name8})}\\
    \sum_{i = 0}^{k-1} p(x)/x^{bi}  (\modu x^b) x^{bi} &+ (p(x)/x^{b k}) x^{b k}
  \end{align*}
\end{proof}
