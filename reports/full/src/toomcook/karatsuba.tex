\section{Karatsuba}
\label{sec:karatsuba}
Karatsuba-algoritmen bygger på ett enkelt knep och visar tydligt varför vi kan
förbättra den asymptotiska tidskomplexiteten jämfört med naiv
polynommultiplikation. Säg att vi vill multiplicera polynomen $p$ och $q$. Vi
kan då dela upp polynomen $p$ och $q$ i två delar så att:
\begin{align*}
  p(x) &= p_0 + p_1 x^{\frac{n}{2}} \\
  q(x) &= q_0 + q_1 x^{\frac{n}{2}}
\end{align*}
Där $n$ är graden på det största polynomet. Multiplikationen kan då skrivas
som:
\begin{align*}
p(x)q(x) &= (p_0 + p_1 x^{\frac{n}{2}})(q_0 + q_1 x^{\frac{n}{2}}) \\
         &= p_0 q_0  + (q_0 p_1 +p_0 q_1 )  x^{\frac{n}{2}} + p_1 q_1  x^n
\end{align*}
Tricket i Karatsuba är följande enkla omskrivning som gör att vi blir av med en
multiplikation så att vi istället för fyra multiplikationer endast behöver tre
distinkta multiplikationer:
\begin{equation*}
  p_0 q_0 + ((p_1 + p_0)(q_1 + q_0) - p_1 q_1 - p_0 q_0)  x^{\frac{n}{2}} + p_1 q_1  x^n
\end{equation*}
Tidskomplexiteten för Karatsuba-algoritmen är då:
\begin{equation*}
  T(n) = 3 T(\lceil n/2\rceil) + cn + d
\end{equation*}
där $n$ är graden på det största polynomet och $c$ och $d$ är konstanter.

Genom att skriva ut rekursionsekvationen får vi att Karatsuba-algoritmen har
den asymptotiska tidskomplexiteten $\Ordo(n^{\log_2 3})$.
