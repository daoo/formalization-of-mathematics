\section{Karatsuba}
Toom-2 är ett specialfall av Toom-Cook och är även känd som
Karatsuba-algoritmen. I Karatsuba-algoritmen delas multiplikanderna upp på
samma sätt som i steg 1.2 i Toom-m algoritmen ovan vilket ger oss de två
polynomen:
\begin{align*}
  u(x) &= u_1 y + u_0 \\
  v(x) &= v_1 y + v_0
\end{align*}
Om vi skall multiplicera polynomen med naiv polynommultiplikation krävs det
fyra multiplikationer:
\begin{equation*}
  u_1 v_1 y^2 + (u_1 v_0+v_1 u_0) y + u_0 v_0
\end{equation*}
Genom att skriva om uttrycket krävs dock bara tre distinkta multiplikationer:
\begin{equation*}
  u_1 v_1 y^2 + ((u_1 + u_0)(v_1 + v_0) - u_1 v_1 - u_0 v_0) y + u_0 v_0
\end{equation*}
Tidskomplexiteten för Karatsuba-algoritmen är då:
\begin{equation*}
  T(n) = 3 T(\ceil{n/2}) + cn + d
\end{equation*}
där n är graden på det största polynomet och c och d är konstanter. Alltså har
Karatsuba-algoritmen den asymptotiska tidskomplexiteten $O(n^{log_2 3})$
