Under senare år har några mycket långa och komplexa matematiska bevis
presenterats. Vissa bevis har också byggt på en mycket omfattande analys av
tusentals fall utförd av datorprogram. Det är mycket tidsödande och svårt, om
ens praktiskt möjligt att för hand kontrollera varje steg ett sådant bevis. Få
matematiker har tid, lust och den speciella kompetensen inom just det specifika
matematiska området som gör att de kan eller vill att ägna år åt att
kontrollera att ett sådant bevis är korrekt. Dessutom finns risken för att ett
fel i ett bevis ändå inte upptäcks vid kontroll om beviset är hundratals- eller
tusentals sidor långt.\cite{harrison2008formal}

Bevisassistenter, datorsystem för att formalisera och verifiera varje logiskt
steg i bevis, kan användas för att kontrollera bevis och därmed öka tilltron
till att de är korrekta och minska risken för att de innehåller fel.

Fyrfärgssatsen\cite{gonthier2008formal} och Feit-Thompsons
sats\cite{aschbacher2004status} är två exempel på satser vars bevis har
formaliserats och kontrollerats i bevisassistenter. För att kunna kontrollera
själva bevisen har också all den matematik som bevisen bygger på formaliserats.
%Lite mer om fyrfärgssatsen här?
%Lite mer om Odd order theorem här?

De verktyg som används vid formalisering och datorverifiering kan även användas
för att verifiera programkod. Formella metoder är således intressant för både
för programmerare och matematiker.
% Niclas: Den här meningen säger väldigt lite. Varför??
Dagens stora och komplexa programvaror skulle kunna utnyttja formella metoder.
%)
Det kan också vara användbart i
kritiska system, till exempel medicinsk utrustning,  där det inte får bli fel
eller i system som inte kan uppdateras i efterhand som hårvarunära mjukvara
på ROM-minnen. De mest använda metoderna idag för
att kontrollera kod bygger på att testa om koden ger korrekt resultat för olika
indata. På detta sätt har man en chans att upptäcka om koden innehåller fel,
men det visar inte att koden saknar fel eftersom det i många fall är omöjligt
att testa alla kombinationer av indata.
% Niclas: + Svårt/Krävande att hitta kända resultat?
Formella metoder skulle i dessa fall
kunna användas till att garantera korrekthet hos koden.

Matematisk programvara som MATLAB spelar en stor roll för beräkningar inom
forskning och industri och det finns därmed ett stort intresse av att de är
% Niclas: känns som det kan finnas bättre ord än buggfria
pålitliga. De är dock inte buggfria. Ett sätt att göra dem mer pålitliga skulle
vara att formalisera de ingående algoritmerna och visa att de är korrekt
implementerade.\cite{mortberg2012}

Vårt projekt går ut på att lära oss formalisering med bevisassistenten \coq och
sedan formalisera och bevisa en matematisk algoritm i \coq.
