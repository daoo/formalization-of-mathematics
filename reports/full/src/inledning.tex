Under senare år har några mycket långa och komplexa matematiska bevis
presenterats. Vissa bevis har också byggt på en mycket omfattande analys av
tusentals fall utförd av datorprogram. Det är mycket tidsödande och svårt, om
ens praktiskt möjligt att för hand kontrollera varje steg ett sådant bevis. Få
matematiker har tid, lust och den speciella kompetensen inom just det specifika
matematiska området som gör att de kan eller vill att ägna år åt att
kontrollera att ett sådant bevis är korrekt. Dessutom finns risken för att ett
fel i ett bevis ändå inte upptäcks vid kontroll om beviset är hundratals- eller
tusentals sidor långt\cite{harrison2008formal}.

Bevisassistenter, datorsystem för att formalisera och verifiera varje logiskt
steg i bevis, kan användas för att kontrollera bevis och därmed minska risken
för att de innehåller fel och öka tilltron till att de är korrekta.

Fyrfärgssatsen\cite{gonthier2008formal} och Feit-Thompsons
sats\cite{aschbacher2004status} är två exempel på satser vars bevis har
formaliserats och kontrollerats i bevisassistenter. Fyrfärgssatsen säger att
varje karta kan färgläggas med fyra färger så att inga regioner med gemensam
gräns får samma färg. Det ursprungliga beviset var över hundra sidor långt och
byggde också datorprogrammerad fallanalys av miljontals fall. Detta gjorde
beviset svårt att kontrollera och kontroversiellt. Beviset formaliserades och
verifierades i bevisassistenten \coq 1995.
%Lite mer om fyrfärgssatsen här?
%Lite mer om Odd order theorem här?

De verktyg som används vid formalisering och datorverifiering kan även användas
för att verifiera programkod. Formella metoder är således intressant för både
för programmerare och matematiker.
% Niclas: Den här meningen säger väldigt lite. Varför??
Dagens stora och komplexa programvaror skulle kunna utnyttja formella metoder.
%)
Det kan också vara användbart i kritiska system, till exempel medicinsk
utrustning, där det inte får bli fel eller i system som inte kan uppdateras i
efterhand som hårdvarunära mjukvara på ROM-minnen.

De mest använda metoderna idag för att kontrollera kod bygger på att testa om
koden ger korrekt resultat för olika indata. På detta sätt har man en chans att
upptäcka om koden innehåller fel, men det visar inte att koden saknar fel
eftersom det i många fall är omöjligt att testa alla kombinationer av indata.
% Niclas: + Svårt/Krävande att hitta kända resultat?
Formella metoder skulle i dessa fall kunna användas till att garantera
korrekthet hos koden.

Ett exempel på projekt för att verifiera kod är CompCert. Det utforskar
möjligheten att utveckla formellt bevisade kompilatorer. Anledningen att man
vill ha en formellt bevisad kompilator är att vid vissa optimeringar så kan
kompilatorn skapa buggar och beräkningsfel. Huvudresultatet av detta är en
fungerande C-kompilator som är bevisad i \coq och som stödjer hela ANSI C (som
är den första standardiserade versionen av C) med få
undantag\autocite{compcert}.

Matematisk programvara som MATLAB spelar en stor roll för beräkningar inom
forskning och industri och det finns därmed ett stort intresse av att de är
% Niclas: känns som det kan finnas bättre ord än buggfria
pålitliga. De är dock inte buggfria. Ett sätt att göra dem mer pålitliga skulle
vara att formalisera de ingående algoritmerna och visa att de är korrekt
implementerade\cite{denes2012refinement}.

Det här projektet går ut på att gruppdeltagarna ska lära sig formalisering med
bevisassistenten \coq och sedan formalisera och bevisa en matematisk algoritm,
\toom, i \coq.

\coq är en av flera avancerade bevisassistenter som det bedrivs aktiv forskning
om. Några andra är Agda, som utvecklats på Chalmers, Z3, som har utvecklats av
Microsoft och kan användas tillsammans med flera stora ickefunktionella språk
som Python, C och .NET och HOL-light, som används av Intel för att bevisa att
vissa hårdvarukomponenter fungerar korrekt och i det pågående projektet att
formalisera Keplers förmodan \cite{hales2008formal}.

\toom är en algoritm för att multiplicera heltal eller polynom. Den är
intressant eftersom den har en bättre asymptotisk tidskomplexitet än
polynommultiplikation utförd direkt enligt definitionen \footnote{För
definition av polynommultiplikation, se
appedix~\ref{appendix:matematikteori}.}, där man multiplicerar varje term i det
ena polynomet med varje term i det andra polynomet vilket har den asymptotiska
tidskomplexiteten $O\left(n^2\right)$\footnote{Funktionen $T(n)$ är $O(f(n))$
om det existerar konstanter $c > 0$ och $n_0 \geq 0$ så att $T(n) \leq c \cdot
f(n)$ för alla $n \geq n_0$.}, där n är graden på det största av de två
polynomen som skall multipliceras.

\section{Rapportens upplägg}
I kapitel~\ref{metod} beskrivs projektets arbetsmetod. I
kapitel~\ref{formalisering} introduceras formalisering av matematik och
bevisassistenter. Kapitel~\ref{coq} beskriver bevisassistenten \coq som har
använts i projektet. Kapitel~\ref{toomcook} definierar algoritmen \toom och ger
ett informellt bevis för att den är korrekt. I kapitel~\ref{formellalg}
presenteras formaliseringen och beviset av \toom i \coq och i
kapitel~\ref{diskussion} diskuteras \coq, formalisering och algoritmen.
