\subsection{Coq}

\subsubsection{Logik (Konstruktiv)}
Tror att det är bättre om någon från matte gör det här

\subsubsection{Curry-Howard isomorphism}
Enligt Curry Howard isomorpismen så är propositioner samma sak som typer och
bevis är samma sak som program. Om vi tar en närmare titt på
funktionsdefinitionen $a \rightarrow b$ så kan vi tolka det som att givet ett
bevis för a så får vi ett bevis för b.
\begin{align*}
  Propotioner &= Typer \\
  Bevis       &= Programs
\end{align*}

\subsubsection{Polymorphism}
En enkel förklaring till polymorphism är att en funktion kan appliceras på
flera olika typer av parametrar. En lite mer komplicerad förklaring är att en
polymorphisk funktion är uppbygd av två olika lambda uttryck där det första har
typer som parametrar och det andra uttrycket har termer som beror av de angivna
typerna som parametrar.
\begin{equation}
  \lambda_{Typer} \rightarrow (\lambda_{Termer} \rightarrow x)
  \label{polymorphsk funktion}
\end{equation}
Detta betyder att vi först skapar en funktion som tar en eller flera
typer som parametrar. Denna yttre funktion ger sedan tillbaka en funktion
där vi har parametrar som nu är bundna till de typerna som vi angav i den
yttre funktionen.

Som de flesta funktionella programmeringspråk så har Coq stöd för ad hoc
polymorphism som även kallas överlagring. Detta innebär att typerna i
funktionen bestämms av sammanhanget. Då coq är ett strikt typat språk används
typklasser för att beskriva vilka typer som är tillåtna. Även då Coq har fullt
stöd för typklasser och överlagring så används detta inte så mycket utan
istället används kanoniska strukturer {\it (Eng Canonical Structures)}.
Kanoniska strukturer är för komplicerade för att ingå i detta kandidatarbete
men de är värda att nämnas då de ofta används för att lösa polymorphism i Coq.

I följande exempel defineras en lista som kan innehålla värden av alla möjliga
typer. Den undre definitionen i exemplet är en rekursiv funktion som lägger till
ett ellement i en lista. Denna funktionen är polymorphisk och fungerar alltså på
alla listor oberoende vilken typ på värden som listan innehåller.
\begin{lstlisting}
Inductive list (X:Type) : Type :=
  | nil  : list X
  | cons : X -> list X -> list X .

Fixpoint append (X:Type) (l : list X) (x : X) : list X :=
  match l with
  | nil       => cons X x (nil X)
  | cons a l' => cons X a (append l' x)
  end.
\end{lstlisting}

\subsubsection{Beroendetyper}
I en polymorphisk funktion så kan en parametertyp bero på vilka typer de
tidigare parametrarna har haft. I beroendetypning så går vi ett steg längre och
låter typerna bero på värdet av en tidigare parameter Nedan finns två
funktionsdefinitioner där beroendetypning används. I exemplena används en klass
som heter Vector och det är inget mer komplicerat en lista med fast längd. I
det första exemplet så gör vi om en lista till en vektor och vektorn ska då ha
samma längd som listan. I det andra exemplet har vi en funktion som tar bort
det första ellementet från en vektor och resultatet blir då en vector med ett
mindre ellement.
\begin{verbatim}
toVec: (list : List) : Vector (lenght list)

removeFirst (v : Vector n) : Vector (n-1))
\end{verbatim}
Det är inte bara funktionella språk som det finns beroendetypade funktioner.
Ett exempel på beroendetypning som de flesta antagligen är bekanta med är
\texttt{printf} i C. Här bestäms antalet parametrar i funktionen av antalet \%
i den första strängen och vilken typ det ska vara på dessa parametrar bestäms
av vilken bokstav som står efter \%-tecknet.
\begin{verbatim}
printf("%s is %d years old and %f.1cm long", name ,age , lenght)
\end{verbatim}

% http://mattam.org/research/publications/Programming_with_Dependent_Types_in_Coq-PPS-260209.pdf

\subsubsection{Grund och taktikspråk}
Coq består av två olika delspråk. Grundspråket kallas Gallina och liknar till
viss del OCaml. Det är i Gallina som de definitioner och funktioner som ska
bevisas skrivs.

Coq innehåller också ett taktikspråk som heter Ltac och innehåller olika
taktiker för att påverka de hypoteser och mål som ska bevisas. Ltac gör det
möjligt att använda samma metoder i Coq som man använder när man skapar ett
pappersbevis. Coq är en interaktiv teorembevisare så när en viss taktik används
uppdateras de hypoteser och mål som ska bevisas och användaren anger då en ny
taktik och detta fortsätter tills alla målen är lösta.
