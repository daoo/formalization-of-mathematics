\section{Abstract}

\subsection{Background}

\subsection{Result}

\subsection{Conclution}



\section{Sammanfattning}

\subsection{Bakgrund}
Både innom mattematiken och datavetenskapen finns ett intresse
för att garrantera korrekthet inom program och bevis. Detta
eftersom buggar i program kan leda till enorma kostnader för
företagen som använder programmet. I bevis kan ett enda litet
slarvfel ta flera år att hitta och om det görs så betyder det
ofta att beviset inte är korrekt och måste göras om.
Bevisassistenten Coq som har en stor del i arbetet är ett
verktyg där vi kan skapa program och bevis där datorn kan
garrantera att de är korrekta.

\subsection{Resultat}
Projektet gick ut på att bevisa \toom -algoritmen vilket gjordes
genom följande steg:
\begin{itemize}
\item 
\end{itemize}


\subsection{Sammanfattning}
Rapporten går ut på att till en början ge läsaren en bild av vad
formalisering och bevisassistenter är samt beskriva utveckling av program
och bevis i coq. Detta följs av en mattematisk beskrivning
och bevisning av \toom. dessa delar knyt sedan ihop genom att
ett bevis för \toom -algoritmen presenteras och förklaras.
