I det här kapitlet beskrivs tillvägagångssättet för formaliseringen av algoritmen.

\section{Litteraturstudie och färdighetsträning}
Som första steg i arbetet genomfördes en studie med mål att projektdeltagarna
skulle få tillräckligt med kunskaper för att kunna programmera och skapa bevis
med hjälp av \coq{} och \ssr{}. Som material användes bland annat
\emph{Software Foundations} av Benjamin C. Pierce som är kursmaterial till en
grundkurs i \coq{}\cite{pierce2012software}, \emph{Coq in a Hurry} av Yves
Bertot som är en kort introduktionsartikel till \coq{}\cite{bertot2006coq} och
\emph{SSReflect tutorial} av Georges Gonthier som är en introduktion till
\ssr{} \cite{gonthier2009ssreflect}. Som en övning i att hantera polynom i \ssr
gjordes även ett bevis av Karatsuba-algoritmen som redan är formaliserad i
\coq{}. I kapitel~\ref{sec:coq} beskrivs \coq.

\section{Definition och bevis för hand}
Först gjordes en informell men detaljerad definition av en generell version av
Toom-Cook. Ett detaljerat bevis gjordes också på papper för senare användning i
implementationen. Resultatet av detta steg i implementationen finns i
kapitel~\ref{sec:toomcook}.

\section{Implementation i \haskell{}}
I samband med framtagningen av den informella definitionen och beviset gjordes
en implementation av Toom-Cook för heltal i \haskell{}. Den gjordes först för
\toomp{3} men generaliserades sedan till \toomp{m}. Implementationen testades
med testfallsgeneratorn QuickCheck genom att jämföra implementationen av
Toom-Cook med den heltalsmultiplikation som finns definierad i \haskell{}.
Resultatet av denna fas i projektet finns i appendix~\ref{sec:haskell}.

Anledningen till att en första implementation gjordes i \haskell{} var att
gruppen inte hade någon erfarenhet av \coq{} innan projektet och man genom att
först implementera en version av algoritmen i \haskell{} som kunde testas
kunde få en bättre förståelse för hur Toom-Cook fungerar.

\section{Implementering och bevis i \coq{}}
När informella definitionen och beviset och den praktiska implementation var
färdiga, utfördes implementeringen av Toom-Cook och dess bevis i \coq{} med
hjälp av \ssr{}. Först skapades en definition av algoritmen i \coq{} med hjälp
av definitionen av algoritmen i \haskell{} och den informella definitionen och
som låg nära dessa definitioner. Denna definition omarbetades senare under
utarbetandet av beviset för att underlätta vissa bevissteg. Det formella
beviset delades som det informella upp i ett huvudbevis, bland annat för att
underlätta arbetsfördelning inom gruppen. Resultatet av detta beskrivs i
avsnitt~\ref{sec:formellimplementation} och avsnitt~\ref{sec:formellbevis}.

\section{Avgränsningar}
Algoritmen i \coq{} har implementerats med de låsta typerna för bland annat
polynom och matriser i \ssr{} som beskrivs i avsnitt~\ref{sec:ssr}. Den kan
alltså användas för att beräkna produkten mellan konkreta polynom. Om tid hade
funnits hade en eller flera exekverbara versioner av Toom-Cook implementerats,
som sedan skulle visats korrekt genom att relatera dem till implementationen
med de låsta typerna (se avsnitt~\ref{sec:ssr} för mer information låsning).
Detta tillvägagångssätt för implementation av matematiska algoritmer diskuteras
i avsnitt~\ref{sec:exekverbar}.
