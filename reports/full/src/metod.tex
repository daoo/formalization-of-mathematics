Följande avsnitt beskriver tillvägångsättet för formalisering av algoritmen.

\section{Litteraturstudie}
Som första steg i arbetet utfördes en studie med mål att få tillräckligt med
kunskaper för att programmera och bevisa med hjälp av \coq{} och \ssr{}. Som
material användes bland annat artiklar, böcker och kursmaterial. Där ibland
\emph{Software Foundations} av Benjamin C. Pierce som är kursmaterial till en
grundkurs i \coq{}, \emph{Coq in a Hurry} av Yves Bertot som är en kort
introduktionsartikel till \coq{} och \emph{SSReflect tutorial} av Georges
Gonthier som är en introduktion till \ssr{}. Som övning utvecklades även bevis av
Karatsuba-algoritmen.

\section{Definition och bevis för hand}
Först gjordes en informell men detaljerad definition av en generell och
abstrakt version av Toom-Cook. Med abstrakt menas utan hänsyn till om det går
att implementera. Ett detaljerat bevis gjordes också på papper för senare
användning i implementationen.

\section{Implementation i \haskell{}}
I samband med framtagningen av pappersbeviset utvecklades en praktiskt
implementation av Toom-Cook för heltal gjordes i \haskell{}. Den byggdes först för
Toom-Cook-3 men generaliserades sedan till Toom-Cook-$n$. Testning gjordes med
QuickCheck genom att jämföra implementationen av Toom-Cook med
heltalsmultiplikationen som finns definierad i \haskell{}. Resultatet av denna
fasen finns i appendix~\ref{app:haskell}.

Anledningen till att en implementation gjordes i \haskell{} var för att vi i det
stadiet inte hade tillräckligt med kunskap om \coq{} för att implementera en ny
algoritm i språket. Eftersom erfarenhet av \haskell{} redan fanns kunde algoritmen
snabbt implementeras och testas, detta gav också en bättre förståelse för hur
Toom-Cook fungerar.

\section{Implementering och bevis i \coq{}}
När pappersbevis och praktisk implementation utfördes implementeringen av
Toom-Cook och dess bevis i \coq{} med hjälp av \ssr{}. Första skapades en
definition av algoritmen i \coq{} väldigt lik den i \haskell{} fast med polynom
istället för heltal. Denna definition omarbetades senare under bevisningen för
att det skulle vara lättare att arbeta med beviset. Uppdelningen av lemmana
gjorde det möjligt Resultatet av detta
beskrivs i avsnitt~\ref{sec:formellimplementation} och
avsnitt~\ref{sec:formellbevis}.

\section{Avgränsningar}
Från början var tanken att en optimerad version av Toom-Cook skulle
implementeras och jämföras med lång multiplikation men i brist på tid gjordes
aldrig detta. Vår implementation kan såldes inte användas, utan vi \emph{vet}
bara att den är korrekt.
