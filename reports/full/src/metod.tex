%Syfte under inledning?
\subsection{Syfte}
Syftet med det här projektet är att formalisera och bevisa korrektheten hos en
generell version av algoritmen Toom-Cook för multiplikation av polynom i \coq/\ssr.

\subsubsection{Avgränsningar}
\emph{(Detta är en kladd!}
Fokus har varit att i första hand få praktiska kunskaper för att kunna använda befintliga
taktiker, typer, funktioner och bevisade resultat i \coq/\ssr för att kunna definiera funktioner och
bevisa resultat om dem. Vi kommer i de flesta fall beskriva hur
de strukturer vi använder fungerar för den praktiska användaren, inte hur de exakt är tekniskt implementerade.
Vi kommer heller inte att implementera en konkret, exekvebar och optimerad algoritm.

-vi fokuserar inte på att verkligen få en snabb algoritm och beräkningskomplexitet nämns bara i förbigående.

- Algoritmen kommer bara implementeras för integritetsområden där det finns interpolationspunkter så att
  interpolationsmatrisen blir inverterbar

- tidskomplexitet hos algoritmen och optimering av algoritmen kommer bara beröras kort

\subsection{Metod}
\emph{(Detta är en kladd!)** betyder att det inte är så bra uttryckt.}
Implementationen av algoritmen **arbetades fram** genom följande steg.
\begin{enumerate}
\item Informell men detaljerad definition av en generell och abstrakt version av algoritmen,
informellt men detaljerat bevis för att den är korrekt. \label{it:1}
\item Implementering av en förenklad men beräkningsbar algoritm. \label{it:2}
\item Implementering och bevis av den generella och abstrakta algoritmen i \textsc{Coq/SSreflect}.\label{it:3}
\end{enumerate}
Steg \ref{it:1} delades upp i följade delsteg.
\begin{enumerate}
 \item Definition av algoritmen \label{it:11}, se avsnitt \ref{in:definition}
 \item Huvudbevis av algoritmens korrekthet, men obevisade lemman \label{it:12}, se prop \ref{prop:1}
 \item Bevis av lemmorna från steg \ref{it:12} \label{it:13}, se lemma \ref{lemma:1} och \ref{lemma:2}
\end{enumerate}

Steg \ref{it:2} gjordes i Haskell av Toom-Cook 3 för heltal. Se appendix \ref{appendix:haskellkod} för resultatet.

Steg \ref{it:3} hade delstegen
\begin{enumerate}
 \item Implementering av algoritmen.
 \item Formalisering av beviset i \ref{it:12}
 \item Formalisering av lemmorna i \ref{it:13} och bevis av ytterligare lemman som det formella beviset krävde.
\end{enumerate}
Resultatet av detta beskrivs i avsnitt \ref{resultat:formell}.

\subsubsection{Övervägningar för metodval.}
\emph{(Detta är en kladd! Det här stycket kommer/borde innehålla: }

Praktiska överväganden:
-Eftersom ingen i gruppen tidigare kunde \coq/\ssr ägnades mycket tid åt att

Metodvalet har motiverats av
-behovet av att lära sig coq samtidigt som algoritmen implementerades
detta gjorde det motiverat att först implementera en föreklad instans av algoritmen för
att kunna testa om den verkade fungera innan man började med den abstrakta och icke-
exekverbara versionen i coq.

- genom att först bevisa huvudsatserna och formulera lemman som behövdes för deras bevis
möjliggjordes arbetsdelning inom gruppen.

- genom att först utarbeta en informell matematisk definition och bevis försäkrar man sig om
att den formella definitionen som modelleras därefter kommer vara sund, vi får möjlighet
att kunna verifiera en struktur för beviset utan att förlora oss i detaljer.
