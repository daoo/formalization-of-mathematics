Definitioner och satser är tagna från [Durbin]

\begin{definition}
En ring är en mängd $R$ tillsammans med två operationer på $R$, nämligen addition + och multiplikation $\cdot$ s.a.

$\bullet$ $a+(b+c) = (a+b)+c$ för alla $a,b,c \in R$.

$\bullet$ Det finns ett neutralt element $0 \in R$ s.a. $a+0=0+a=a$ för varje $a \in R$.

$\bullet$ För varje $a \in R$ finns ett element $-a \in R$ s.a. $a + (-a) = (-a) + a = 0$.

$\bullet$ $a + b = b + a$ för alla $a,b \in R$.

$\bullet$ $a \cdot (b \cdot c)=(a \cdot b) \cdot c$ för alla $a,b,c \in R$.

$\bullet$ $a \cdot (b + c) = a \cdot b + a \cdot c$ för alla $a,b,c \in R$.

\noindent
Det neutrala elementet $0$ kallas för nolla.

\end{definition}

\noindent\textbf{Exempel.} Betrakta mängden av alla 2x2-matriser vars element är reella tal. Mängden tillsammans 
med operationerna matrisaddition och matrismultiplikation bildar en ring, där det neutrala elementet är $
\begin{pmatrix}
 1 & 0 \\
 0 & 1
\end{pmatrix}.
$

\begin{definition}
 En ring R sägs vara kommutativ om $a \cdot b = b \cdot a$ för alla $a,b \in R$.
\end{definition}

\begin{definition}
 Ett element e i R sägs vara ett multiplikativ enhetselement till en ring om $e \cdot a = a \cdot e = a$ för varje $a \in R$.
\end{definition}

\begin{definition}
 Ett element $a \neq 0$ i en kommutativ ring R sägs vara en nolldelare i R om det finns ett elementet
$b \neq 0$ s.a. $a \cdot b = 0$.
\end{definition}

\begin{definition}
 En kommutativ ring med multiplikativ enhetselement $e \neq 0$ och inga nolldelare sägs vara ett integritets område.
\end{definition}

\begin{definition}
Låt R vara en kommutativ ring. Ett polynom i variablen x över R är ett uttryck på formen 

\begin{align*}
 a_0+a_1x+a_2x^2+...+a_nx^n
\end{align*}
där koefficienterna $a_0, a_1,...,a_n$ är element i R. Om $a_n \neq 0$, då är heltalet n graden av polynomet, och $a_n$ är dess ledande koefficient. Två polynom i x är lika om och endast om 
koefficienterna av samma grad är lika. Mängden av alla polynom i x över R beteckans R[x]. 

\end{definition}

\begin{definition}
 Låt
\begin{align*}
 p(x)=a_0+a_1x+...+a_mx^m
\end{align*}
och
\begin{align*}
 q(x)=b_0+b_1x+...+b_nx^n
\end{align*}
vara polynom över en kommutativ ring R. Då
\begin{align*}
 p(x)q(x)=a_0b_0+(a_0b_1+a_1b_0)x+...+a_mb_nx^{x+n}
\end{align*}
koefficienten av $x^k$ är
\begin{align*}
 a_0b_k+a_1b_{k-1}+a_2b_{k-2}+...+a_kb_0
\end{align*}

\end{definition}

\begin{theorem}
 Om R är ett integritetsområde, då är R[x] ett integritetsområde.
\end{theorem}

Som följd av den här satsen är även R[x][y] ett integritetsområde.
