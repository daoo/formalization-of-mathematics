\begin{definition}
En ring är en mängd $R$ tillsammans med två operationer på $R$, nämligen
addition + och multiplikation $\cdot$ s.a.

$\bullet$ $a+(b+c) = (a+b)+c$ för alla $a,b,c \in R$.

$\bullet$ Det finns ett neutralt element $0 \in R$ s.a. $a+0=0+a=a$ för varje $a \in R$.

$\bullet$ För varje $a \in R$ finns ett element $-a \in R$ s.a. $a + (-a) = (-a) + a = 0$.

$\bullet$ $a + b = b + a$ för alla $a,b \in R$.

$\bullet$ $a \cdot (b \cdot c)=(a \cdot b) \cdot c$ för alla $a,b,c \in R$.

$\bullet$ $a \cdot (b + c) = a \cdot b + a \cdot c$ för alla $a,b,c \in R$.

\noindent
Det neutrala elementet $0$ kallas för nolla.
\end{definition}

\noindent\textbf{Exempel.} Betrakta mängden av alla 2x2-matriser vars element
är reella tal. Mängden, som betecknas
$M_2(\mathbb{R})= \{
\begin{pmatrix}
  a & b\\
  c & d 
\end{pmatrix}
: \; a,b,c,d \in \mathbb{R} \}$, tillsammans med operationerna matrisaddition
och matrismultiplikation bildar en ring, där det neutrala elementet är $
\begin{pmatrix}
 1 & 0 \\
 0 & 1
\end{pmatrix}.
$

\begin{definition}
En ring R sägs vara kommutativ om $a \cdot b = b \cdot a$ för alla $a,b \in R$.
\end{definition}

\begin{definition}
Ett element e i R sägs vara en enhet till en ring om
$e \cdot a = a \cdot e = a$ för varje $a \in R$.
\end{definition}

\begin{definition}
Ett element $a \neq 0$ i en kommutativ ring R sägs vara en nolldelare i R om
det finns ett elementet $b \neq 0$ s.a. $a \cdot b = 0$.
\end{definition}

\begin{definition}
En kommutativ ring med unity $e \neq 0$ och inga nolldelare sägs vara ett
integritets område.
\end{definition}
