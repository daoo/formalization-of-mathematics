\subsection{Formalisering av matematik}
Vanlig matematisk text är en blandning mellan formler och naturligt språk. Ett
matematiskt bevis i en kursbok eller vetenskaplig artikel kan sägas vara en
skiss, ett argument, som ska övertyga läsaren om satsen är korrekt och beviset
är giltigt. Varje litet logiskt bevissteg behöver inte tas med, särskilt inte
om den avsedda läsaren är van vid typen av bevis det handlar om.

Man låter också många saker framgå av sammanhanget. Symbolen 1 kan till exempel
stå för bland annat det naturliga talet 1 som är efterföljare till 0, det
rationella talet $\frac{1}{1}$ eller det multiplikativa enhetselementet i en
grupp. Oftast är det inga problem att ur sammanhanget förstå vilken betydelse
av 1 som avses.
%fast coq kan ju också göra typinferens

När är ett bevis giltigt? En definition är att
\begin{quote}
... the correctness of a mathematical text is verified by comparing it, more or
less explicitly, with the rules of a formalized language\cite{bourbaki}.
\end{quote}

Så för att ett datorprogram mekaniskt ska kunna kontrollera ett bevis måste det
\emph{formaliseras}. Sådant som för en mänsklig läsare framgår ur sammanhanget
måste göras explicit. Bevissteg som är så små så att de ses som självklara
måste också formuleras.

Datorsystemet kontrollerar sedan om varje steg i beviset följer från föregående
steg och från axiom genom fastslagna \emph{härledningsregler}. Dessa är enkla
logiska regler om hur man får gå från givna premisser till slutsatser. Till
exempel säger regeln \emph{modus ponens} att man ur förutsättningarna $A$ och
$A \to B$ får dra slutsatsen att $B$ gäller. Den kontrollerar också om de
uttryck man skriver in är tillåtna.
$\forall 3^{\frac{+}{\in}} \leftrightarrow =$ är till exempel inget välformat
uttryck, även om de ingående symbolerna är matematiska och logiska symboler.

För att formalisera matematik måste man alltså bestämma vilka härledningsregler
som ska vara tillåtna, vilka axiom man skall ha, vilka symboler som ska vara
tillåtna att skriva och vilka uttryck som är godkända. Eftersom det finns olika
möjliga val finns det olika \emph{formella språk} eller \emph{logiska system}.

De flesta logiska system som används för att formalisera matematik kan dock
uttrycka och härleda ungefär samma saker, möjligen på något olika sätt,
eftersom man har varit intresserad av att fånga och beskriva naturliga logiska
resonemang.****

En viktig skillnad är dock den mellan intuitionistisk och klassisk logik.
\emph{Intuitionistisk typteori}\cite{martin1984intuitionistic} är grunden i
\emph{Coq/Ssreflect}\cite{bertot2004interactive}. Det går att se den som en
bevisbarhetslogik. I klassisk logik är det en logisk sanning att för alla
satser $A$ gäller $A \lor \neg A$. Så att om $A$ inte är sann, så måste
$\neg A$ vara sann. Men i intuitionistisk logik låter vi $A$ är sann betyda
``det finns ett bevis för $A$''. Och bara för att vi inte har något bevis för
$A$ betyder det inte att vi har ett bevis för $\not A$. Detta betyder att vissa
motsägelsebevis som är giltiga i klassisk logik inte är giltiga i
intuitionistisk logik. \cite{proofdependent}
