\coq är en \emph{bevisassistent} som används för att formalisera matematik och
kod och formalisera bevis för att den är korrekt. I det här avsnittet ges en
informell introduktion till vad ett matematiskt bevis är och vad som menas med
formalisering av matematik. Sedan beskrivs vad en bevisassistent är och kan
göra.

\section{Matematiska bevis och datorverifiering}
Vanlig matematisk text är en blandning mellan formler och naturligt språk. Ett
matematiskt bevis i en kursbok eller vetenskaplig artikel kan sägas vara en
skiss av ett fullständigt bevis, ett argument, som ska övertyga läsaren om att
satsen är korrekt och beviset är giltigt. Varje litet logiskt bevissteg behöver
inte tas med, särskilt inte om den avsedda läsaren är van vid typen av bevis
som det handlar om. Dessa steg kan läsaren själv fylla i.

%Exempel andragradsekvation?

Det är också vanligt att man låter många saker framgå av sammanhanget. Till
exempel kan symbolen 1 kan stå för olika saker, bland annat det naturliga talet
1 som är efterföljare till 0, det rationella talet $\frac{1}{1}$ eller det
multiplikativa enhetselementet i en ring\footnote{För en definition av ringar,
se appendix \ref{appendix:matematikteori}}. Oftast kan en mänsklig läsare ur
sammanhanget förstå vilken betydelse av 1 som avses i ett visst fall.
%fast coq kan ju också göra typinferens

För att ett datorsystem ska kunna kontrollera bevis krävs att det finns en
exakt definition av vad det innebär att ett bevis är giltigt. En definition är
att
\begin{quote}
``... the correctness of a mathematical text is verified by comparing it, more or
less explicitly, with the rules of a formalized language'' -- \cite{bourbaki1968sets}.
\end{quote}

Givet den definitionen måste ett bevis \emph{formaliseras} för att ett
datorsystem mekaniskt ska kunna kontrollera det. Sådant som en mänsklig läsare
förstår ur sammanhanget måste göras explicit. Bevissteg som är så små att de
ses som självklara för en människa måste också skrivas ut.

Datorsystemet kan sedan kontrollera om varje steg i beviset följer från
föregående steg eller från axiom genom en bestämd mängd fastslagna
\emph{härledningsregler}. Dessa är enkla logiska regler om hur man får gå från
givna premisser till slutsatser. Till exempel säger härledningsregeln
\emph{modus ponens} att man ur förutsättningarna $A$ och $A \to B$ får dra
slutsatsen att $B$ gäller. Systemet kontrollerar också om de uttryck man
skriver in är tillåtna, det vill säga om man använt rätt syntax.
$\forall 3^{\frac{+}{\in}} \leftrightarrow =$ är till exempel inget välformat
uttryck. Det har ingen matematisk betydelse även om de ingående symbolerna är
matematiska och logiska symboler. Ett annat exempel på felaktig syntax eller
felformade uttryck är om man försöker definiera en funktion
\begin{lstlisting}
 exempel_function (n : nat) : bool := n + 2.
\end{lstlisting}
eller med matematisk notation
\begin{align*}
  exempel\_function :\ &\mathbb{N} \to \{sant, falskt\} \\
                       &n \mapsto n + 2
\end{align*}
som enligt specifikationen ska gå från naturliga tal (\C{nat}) till boolska
sanningsvärden (\C{bool}) men som samtidigt sägs ska anta värdet $n + 2$ för
varje naturligt tal $n$.

För att formalisera matematik i en bevisassistent måste man alltså bestämma de
exakta reglerna som bevisassistenten ska följa:
\begin{itemize}
  \item vilka härledningsregler som ska vara tillåtna,
  \item vilka axiom som skall användas,
  \item vilka symboler det är tillåtet att forma uttryck med,
  \item och vilka uttryck av dessa symboler som är godkända.
\end{itemize}
Eftersom det finns olika möjliga val för alla dessa punkter finns det olika
\emph{formella språk} eller \emph{logiska system}. Ett val av en uppsättning
härledningsregler, axiom och regler för vilka symboler och uttryck som är
tillåtna ger oss \emph{ett} möjligt logiskt system.

De flesta logiska system som används för att formalisera matematik kan dock
uttrycka och härleda ungefär samma saker, möjligen på något olika sätt,
eftersom skaparna har varit intresserade av att fånga och beskriva naturliga
logiska resonemang.

En viktig skillnad är dock den mellan intuitionistisk och klassisk logik. En
utvidgning av \emph{intuitionistisk typteori}\cite{martin1984intuitionistic} är
det logiska system som finns i grunden till
\coq/\ssr\cite{bertot2004interactive}. Det går att se den som en
bevisbarhetslogik. Skillnaden mellan den och klassisk logik kan illustreras
genom följande exempel: I klassisk logik är det en logisk sanning att $A \lor
\neg A$ gäller för alla satser $A$ \footnote{Detta brukar kallas lagen om det
uteslutna tredje.}. Så om $A$ inte är sann måste $\neg A$ vara
sann\cite{bennet2004forsta}.

I intuitionistisk logik däremot låter vi ``$A$ är giltig'' betyda ``det finns
ett bevis för $A$''. Om $A$ inte är giltig finns det alltså inget bevis för
$A$. Men bara för att det inte finns något bevis för $A$ betyder det inte att
det i stället nödvändigtvis finns ett bevis för $\neg A$. Detta betyder att
vissa motsägelsebevis som är giltiga i klassisk logik inte kommer vara giltiga
i intuitionistisk logik\cite{barendregt2001proofdependent}.

Om vi i klassisk logik har antagit att $\neg A$ gäller och visat att detta
leder till en motsägelse så kan vi, eftersom då $A \lor \neg A$ är en logisk
sanning måste minst en av $A$ och $\neg A$ måste gälla, därmed dra slutsatsen
att $A$ gäller.

\section{Bevisassistenter}
Ovan diskuteras vad som krävs för att ett datorsystem ska kunna kontrollera
giltigheten i matematiska bevis. De tidigaste bevissystemen för datorer kunde
bara göra detta. Det innebär att användaren själv måste skriva in varje enskilt
logiskt steg i det formella beviset, medan datorsystemet bara passivt
kontrollerar att dessa är giltiga.

Senare system ger användaren möjligheten att i stället skriva ett
\emph{bevisskript}, en rad instruktioner till systemet om hur det formella
beviset ska byggas upp, om vilka taktiker systemet ska använda, utan att själv
behöva göra varje enskilt steg. Bevisassistenten bygger steg för steg det
formella beviset och kontrollerar att instruktionerna från användaren ger
upphov till ett korrekt bevis. En större eller mindre grad av automatisering är
också möjlig. Bevisassistenter kan utformas så att användaren kan ange att
vissa steg i beviset ska lösas automatiskt eftersom de är så enkla att
systemet själv kan hitta de härledningsregler och axiom som krävs för att visa
dem.

Helt automatiserade bevismaskiner finns också. Då behöver användaren bara
formulera ett antagande, och systemet söker sedan själv efter ett bevis för
detta. De klarar dock ofta inte av att hitta komplicerade matematiska bevis
inom skiljda matematiska områden\cite{geuvers2009proof}.
%Sista meningen är lite luddig, för jag vet egentligen inte vad som gäller.
