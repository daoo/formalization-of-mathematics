\coq{} är en \emph{bevisassistent} som används för att formalisera och bevisa
matematik och kod.
% Denna meningen känns lite märklig jag antar "och formalisera bevis för att
% den är korrekt är kopplad till kod, men det är inte så tydligt //J
I det här avsnittet ges en introduktion till vad ett matematiskt bevis är och
vad som menas med formalisering av matematik. Sedan beskrivs vad en
bevisassistent är och kan göra.

\section{Matematiska bevis och datorverifiering}
Vanlig matematisk text är en blandning mellan formler och naturligt språk. Ett
matematiskt bevis i en kursbok eller vetenskaplig artikel kan sägas vara en
skiss av ett fullständigt bevis, ett argument, som ska övertyga läsaren om att
satsen är korrekt och beviset är giltigt. Varje litet logiskt bevissteg behöver
inte tas med, särskilt inte om den avsedda läsaren är van vid typen av bevis
som det handlar om. Dessa steg kan läsaren själv fylla i.

%Exempel andragradsekvation?

Det är också vanligt att många saker får framgå av sammanhanget.
% Kan man formulera det annorlunda än "låter många saker"? //J
Till exempel kan symbolen 1 stå för olika saker, bland annat det naturliga
talet 1 som är efterföljare till 0, det rationella talet $\frac{1}{1}$ eller
det multiplikativa enhetselementet i en ring\footnote{För en definition av
ringar, se appendix~\ref{sec:algebra}}. Oftast kan en mänsklig läsare ur
sammanhanget förstå vilken betydelse av 1 som avses i ett visst fall.
%fast coq kan ju också göra typinferens

För att ett datorsystem ska kunna kontrollera bevis krävs att det finns en
exakt definition av vad det innebär att ett bevis är giltigt. En definition är
att
\begin{quote}
``... the correctness of a mathematical text is verified by comparing it, more or
less explicitly, with the rules of a formalized language'' -- \cite{bourbaki1968sets}.
\end{quote}
% TODO: Översätta citatet?

Givet den definitionen måste ett bevis \emph{formaliseras} för att ett
datorsystem mekaniskt ska kunna kontrollera det. Det innebär att sådant som en
mänsklig läsare förstår ur sammanhanget måste göras explicit och bevissteg som
är så små att de ses som självklara för en människa måste också skrivas ut.

Datorsystemet kan sedan kontrollera om varje steg i beviset följer från
föregående steg eller från axiom genom en bestämd mängd fastslagna
\emph{härledningsregler}. Dessa är enkla logiska regler om hur man får gå från
givna premisser till slutsatser. Till exempel säger härledningsregeln
\emph{modus ponens} att man ur förutsättningarna $A$ och $A \to B$ får dra
slutsatsen att $B$ gäller. Systemet kontrollerar också om de uttryck man
skriver in är tillåtna, det vill säga om man använt rätt syntax.
$\forall 3^{\frac{+}{\in}} \leftrightarrow =$ är till exempel inget välformat
uttryck. Det har ingen matematisk betydelse även om de ingående symbolerna är
matematiska och logiska symboler. Ett annat exempel på felaktig syntax eller
felformade uttryck är om man försöker definiera en funktion
\begin{lstlisting}
exempel_funktion (n : nat) : bool := n + 2.
\end{lstlisting}
eller med matematisk notation
\begin{align*}
  exempel\_funktion :\ &\mathbb{N} \to \{sant, falskt\} \\
                       &n \mapsto n + 2
\end{align*}
som enligt specifikationen ska gå från naturliga tal (\C{nat}) till boolska
sanningsvärden (\C{bool}) men som samtidigt sägs ska anta värdet $n + 2$ för
varje naturligt tal $n$.

För att formalisera matematik i en bevisassistent måste man alltså bestämma de
exakta reglerna som bevisassistenten ska följa:
\begin{itemize}
  \item vilka härledningsregler som ska vara tillåtna,
  \item vilka axiom som skall användas,
  \item vilka symboler det är tillåtet att forma uttryck med,
  \item och vilka uttryck av dessa symboler som är godkända.
\end{itemize}
Eftersom det finns olika möjliga val för alla dessa punkter finns det olika
\emph{formella språk} eller \emph{logiska system}. Ett val av en uppsättning
härledningsregler, axiom och regler för vilka symboler och uttryck som är
tillåtna ger oss \emph{ett} möjligt logiskt system.

De flesta logiska system som används för att formalisera matematik kan dock
uttrycka och härleda ungefär samma matematik och logik, möjligen på något olika
sätt, eftersom skaparna har varit intresserade av att fånga och beskriva
naturliga logiska resonemang.

En viktig skillnad är dock den mellan intuitionistisk och klassisk logik. En
utvidgning av \emph{intuitionistisk typteori}\cite{martin1984intuitionistic} är
det logiska system som finns i grunden till \coq{}\cite{bertot2004interactive}.
Det går att se den som en bevisbarhetslogik. Skillnaden mellan den och klassisk
logik kan illustreras genom följande exempel: I klassisk logik är det en logisk
sanning att $A \lor \neg A$ gäller för alla satser $A$ \footnote{Detta brukar
kallas lagen om det uteslutna tredje.}. Så om $A$ inte är sann måste $\neg A$
vara sann\cite{bennet2004forsta}.

I intuitionistisk logik däremot låter vi ``$A$ är giltig'' betyda ``det finns
ett bevis för $A$''. Om $A$ inte är giltig finns det alltså inget bevis för
$A$. Men bara för att det inte finns något bevis för $A$ betyder det inte att
det i stället nödvändigtvis finns ett bevis för $\neg A$. Detta betyder att
vissa motsägelsebevis som är giltiga i klassisk logik inte kommer vara giltiga
i intuitionistisk logik\cite{barendregt2001proofdependent}.

Om vi i klassisk logik har antagit att $\neg A$ gäller och visat att detta
leder till en motsägelse så kan vi, eftersom då $A \lor \neg A$ är en logisk
sanning måste minst en av $A$ och $\neg A$ måste gälla, därmed dra slutsatsen
att $A$ gäller.

\section{Bevisassistenter}
Ovan diskuteras vad som krävs för att ett datorsystem ska kunna kontrollera
giltigheten i matematiska bevis. De tidigaste bevissystemen för datorer kunde
endast göra detta så användaren var själv tvungen att skriva in varje enskilt
logiskt steg i det formella beviset, medan datorsystemet bara passivt
kontrollerade att dessa var giltiga.

Senare system ger användaren möjligheten att i stället skriva ett
\emph{bevisskript}, som är en rad instruktioner till systemet som talar om hur
det ska bygga upp det formella beviset, utan att användaren själv behöver mata
in varje enskilt logiskt steg.
% Vad menas med taktiker, har vi förklarat tactic tidigare? //J
% Du har rätt, jag försöker få in det.
Instruktionerna i bevisskriptet kan ha formen av \emph{taktiker}, som säger till
bevisassistenten vilken form av bevisstrategi som ska användas under olika steg
i beviset. Till exempel kan användaren instruera bevisassistenten om att
beviset ska göras med taktiken induktion över någon parameter i satsen.

Användaren kan skriva bevisskriptet interaktivt. Då ger användaren
taktikinstruktionerna till systemet en i taget. Efter varje instruktion
kontrollerar systemet om taktiken gick att genomföra, och efter varje genomförd
taktik uppdaterar systemet informationen som visas för användaren om var i
beviset man nu befinner sig och vad som återstår att bevisa. Bevisassistenten
bygger på detta sätt steg för steg upp det formella beviset efter
instruktionerna från användaren och kontrollerar efter hand att de ger upphov
till ett korrekt bevis.
% Typcheckning av bevisobjektet finns också i vissa bevisassistenter, men jag vet inte
% om det är så i alla.
Bevisassistenter kan utformas så att användaren kan ange att vissa steg i
beviset ska lösas automatiskt eftersom de är så enkla att systemet själv kan
hitta de härledningsregler och axiom som krävs för att visa dem.

% Kan man få in ordet interaktiv och mer tydligt förklara vad det innebär
% att en bevisassistent är interaktiv. //J Nu har jag försökt. Å

Helt automatiserade bevismaskiner finns också. Då behöver användaren bara
formulera ett antagande, och systemet söker sedan själv efter ett bevis för
detta. De klarar dock ofta inte av att hitta komplicerade matematiska bevis
inom skiljda matematiska
områden\cite{geuvers2009proof}\cite{gonthier2009ssreflect}.
%Sista meningen är lite luddig, för jag vet egentligen inte vad som gäller.
