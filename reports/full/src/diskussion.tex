Från början tänkte vi att vi skulle implementera specialiserade versioner av
Toom-Cook, bevisa dem, och testa deras prestanda. Men arbetets komplexitet
hindrade oss från att hinna med detta.

\section{Är det realistiskt att formalisera bevis och kod?}
Implementationen av algoritmen och beviset i Coq var mycket tidskrävande, även
för en relativt liten algoritm med ett kortare matematiskt bevis. Flera veckors
arbete med flera personer behövdes för att färdigställa beviset. Jämförelsevis
tog implementationen i Haskell samt testning med QuickCheck tog mindre än en
dag. Frågan är om detta är en rättvis jämförelse eftersom vi hade mycket mer
erfarenhet av Haskell än Coq. Även om mer erfarenhet skulle ha kortat ner
utvecklingstiden markant tar formella bevis flera storleksordningar längre tid
att utveckla än att implementera algoritmen i till exempel Haskell.

Erfarenhet är inte den enda lösningen, Coq är ett forskningsprojekt och lider
av diverse problem. Till exempel interaktionsproblem med programvaran, de
grafiska gränssnitten som finns har en del buggar och saknar många funktioner
Som finns i andra moderna utvecklingsmiljöer för andra språk. Vidare är
dokumentationen för standardbiblioteket väldigt dålig eller obefintlig. När man
sitter och bevisar går väldigt mycket tid går åt till att enbart sitta och söka
i källkoden för att hitta rätt lemma. Om man kan lösa dessa problemen skulle
tiden som behövs kortas ner avsevärt.

Att använda Coq för att bevisa kod är kanske i dagsläget bara relevant om det
är väldigt viktigt att koden är korrekt. För matematiker kan det vara ett mer
relevant verktyg eftersom Coq stoppar direkt när man gjort fel istället för små
enkla fel smyger sig igenom hela beviset.

\section{Är beroendetypning användbart?}
Beroendetypning är bra, men innan man kan motivera beroendetypning behöver man
veta vad som är bra med ett typsystem. Typer specificerar restriktioner för vad
kod kan göra, kompilatorn kan sedan se till att dessa restriktioner följs. Det
kan alltså ses som ett verktyg som hjälper till att få koden korrekt. Eftersom
typer är en slags dokumentation tvingas också programmeraren att tänka på
\emph{vad} det är koden ska göra istället för \emph{hur} detta ska göras. Det
ses som positivt då många buggar uppkommer då kod skrivs utan att ha koll på
vad den ska göra. Denna motivering gäller generellt för typsystem och kan syfta
på till exempel Haskell. Men det gäller även för beroendetypade system då
beroendetypning är ett generellare system.

Beroendetypning är bra, matrismultiplikation

Man slipper tänka, kompilatorn säger ifrån

\section{Coq på Chalmers}

\section{Mer}
Kan vi starta företag?
