\section{Kort introduktion till programmering i \coq och Utvecklingsmiljön \coq
IDE}

\subsection{Interaktiv beskrivning av \coq IDE}

\begin{figure}[H]
  \centering
  \includegraphics[width=\textwidth]{images/Overview}
  \caption[Översikt av \coq IDE]
   {Översikt av de olika delarna i \coq IDE}
\end{figure}

\begin{enumerate}
\item Textredigerare. I den här rutan skriver användaren sina program och bevis
\item Textfönster för mål och kontext. Här visas vilka mål man vill uppnå och
  vilka värden man för närvarande har i kontext. Dessa uppdateras efter varje
  utförd taktik.
\item Textfönter för meddelanden. Här dyker felmeddelanden, svar på
  gjorda sökningar och övrig information.
\item Symboler för att stega framåt eller bakåt i koden. När vi stegar framåt
  så evalueras koden som stegas förbi.


\begin{figure}[h!]
  \centering
  \includegraphics[width=150mm]{images/Kontext}
  \caption[Fönster för kontext och mål]
   {Textfönster för kontext och mål. Detta är en förstoring av (2) från
     föregående figur}
\end{figure}

Förstoring av (2)
\item Kontext, här visas vilka hypoteser och variabler som vi för tillfället
  har i beviset.
\item Mål, här visas vilka mål som ska uppnås. Det översta målet är det som
  användaren arbetar med för tillfället och det är det målet som kommer att
  påverkas av nästkommande taktik.
\end{enumerate}

\subsection{Evaluering av kod}
\coq är uppbygtt av satser där varje sats avslutas med en punkt. Koden
evalueras sedan en sats i taget och de delar av koden som har blivit evaluerade
markeras med en grön färg och det går inte längre att göra några ändringar i
dessa. Om man skulle vilja göra en ändring så får man i så fall stega tillbaks
i programmet och göra ändringen.

\subsection{Bevis}
Det finns flera olika nyckelord för att starta ett bevis bland annat
\C{theorem} och \C{lemma}. Det är ingen skillnad på vilket av dessa orden man
använder för att påbörja ett bevis utan de är bara till för att användaren ska
kunna gradera sina bevis. För enkelhetens skull kommer bara nyckelordet
\C{lemma} användas i fortsättningen. Ett bevis i \coq är uppbyggt på följande
sätt

\begin{lstlisting}
Lemma bevisnamn : påstående som ska bevisas.
Proof.
  taktiker.
Qed.
\end{lstlisting}

Här är ett exempel på ett bevis för att följande påstående är en tautologi.

\begin{figure}[H]
  \centering
  \includegraphics[width=100mm]{images/Proof_part1}
  \caption[Exempel på bevis i \coq]
   {Exempel på ett bevis för en tautologi i \coq}
\end{figure}

Om vi nu stegar igenom beviset så kan vi se att både målet och kontexten
förändras efter varje steg

\begin{figure}[H]
  \centering
  \includegraphics[width=150mm]{images/Proof_part2}
  \caption[Bevis i \coq IDE]
   {Vi har nu flyttat hypoteserna A medför B, B medför C och A
    från målet till kontexten och med taktiken \C{move}}
\end{figure}

\begin{figure}[H]
  \centering
  \includegraphics[width=150mm]{images/Proof_part3}
  \caption[Bevis i \coq IDE]
   {Vi har nu använt oss av Hypotesen B medför C och vi
    kan se att målet nu har ändrat sig från A till B}
\end{figure}

När alla mål är bevisade så talar man om att beviset är klart genom att skriva
\C{Qed.} vilket översatt till svenska betyder "Vilket skulle bevisas".

\section{Variabler och Funktioner}

Det går att definiera variabler i \coq med nyckelordet \C{Varible} När en
variabel deklareras på global nivå kan den sedan användas som parameter i bevis
och funktionsdefinitioner utan att behöva ange typen. En variabel kan också ses
som en hypotes men mer om detta i stycket om Curry Howard isomorphism.

För att definiera funktioner och konstanter så används nyckelordet
\C{Definition} som sedan sedan följs av funktions namn, parametrar och
funktionskropp.

\coq tillåter bara rekursion som är garanterad att avsluta. Det vill säga att
någon av parametrarna i det rekursiva kallet måste närma sig basfallet eller så
måste användaren ange ett bevis för att funktionen är garanterad att avsluta.
Så istället för att använda \C{Define} för att definiera en funktion så måste
man använda \C{Fixpoint} istället. Det skulle till exempel inte gå att göra en
rekursiv funktion för Collatz problem i \coq. Detta eftersom parametern bara
minskar om talet är jämt och det inte går att bevisa att basfallet alltid nås.

Collatz problem.
\begin{equation}
T(n) = \left\{\begin{matrix} n/2, & \mbox{om }n\equiv0\mbox{ (mod 2)} \\ 3n+1,
                         & \mbox{om }n\equiv1\mbox{ (mod 2)} \end{matrix}\right.
\end{equation}

\section{Mönstermatchning}
Mönstermatchning i \coq påminner till stor del om \C{case} i Haskell.
Skillnaden är att istället för att skriva \C{case x of} som i Haskell så skrivs
\C{match x with} där x är namnet på variabeln ska matchas i båda fallen. I \coq
så måste även alla efterföljande rader utan den första börja med symbolen $|$.
Matchningssatsen måste även avslutas med \C{end}.

\begin{figure}[H]
  \centering
  \includegraphics[width=150mm]{images/Variables_and_Functions}
  \caption[Variabler och funktioner]
   {Exempel på hur variabler och funktioner definieras i \coq}
\end{figure}
