\section{\ssr}
\ssr är ett tillägg till Coq som utvecklats av Georges Gonthier.
Namnet kommer från small-scale reflection (småskalig reflektion) vilket är en typ av
bevismetodologi. Denna typ av bevismetodologi lämpar sig särskilt väl för att
formalisera långa matematiska bevis.
En central del i small-scale reflection är att arbeta med olika men ekvivalenta
representationer,
bland annat kan till exempel boolsk logik användas för att resonera
kring och ``räkna'' med motsvarande propositioner, så kallad ``boolean
reflection''.

\ssr har ett stort bibliotek med matematiska satser som är formellt bevisade,
bland annat finns stora delar av linjär algebra och viktiga resultat inom
ändlig gruppteori formaliserade. Namnsystemet för satser är även utformat på ett
systematiskt sätt så att det går smidigt att söka i biblioteket, med viss
erfarenhet går det att gissa sig till namnet på en viss sats.

Vad gäller själva taktikspråket introducerar \ssr bara tre nya tactics men
utökar funktionaliteten i ett antal redan existerande tactics så att en tactic
i \ssr kan svara mot flera tactics i Coq, ett exempel på detta är
rewrite-tacticen som i \ssr utgör ett enat gränssnitt för omskrivningar,
expansion av definitioner och partiell evaluering \cite{gonthier2008small},
vilket kräver flera separata tactics i Coq.
I slutändan har vi färre tactics att hålla reda på i \ssr. Utöver rent funktionella
förändringar tillhandahåller \ssr även verktyg för en bättre layout och
struktur av bevisen.

En del definitioner är låsta i \ssr, det gäller till exempel definitionerna av
matriser och polynom. Låset på definitionerna är till för att undvika att
expandera definitionerna när vi försöker förenkla uttryck då det kan leda till
tunga typchecknings-beräkningar som tar mycket lång tid. En konsekvens av att
definitionerna är låsta är att vi inte kan utföra beräkningar med dem.
