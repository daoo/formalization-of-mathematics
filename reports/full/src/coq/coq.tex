Formalisering är ett aktivt forskningsområde och det finns flera datorverktyg
för formalisering och verifiering av formella bevis som är under aktiv
utveckling. Det här projektet använder verktyget \coq. \coq är ett
beroenedetypat funktionellt programmeringsspråk som i vissa avseenden liknar
det välkända funktionella programmeringsspråket Haskell. Men till skillnad från
Haskell är \coq även en \emph{interaktiv bevisassistent} som kan användas till
att utveckla formella bevis. Program och bevis som är skrivna i \coq går att
exportera till programspråken Haskell, OCaml and Scheme vilket gör att man kan
skriva och bevisa de mest kritiska delarna i ett program i \coq och sedan
exportera dem och köra dem tillsammas med icke bevisad kod i till exempel
Haskell.

\section{Polymorfi}
En kort och enkel förklaring till polymorfi är att en funktion kan appliceras
på flera olika typer av parametrar. En korrektare förklaring till polymorfi är
att en polymorfisk funktion består av två olika delfunktioner. Den första av
dessa funktioner tar typer som parametrar och returnerar den andra
delfunktionen vars parametertyper beror på parametrarna i den första
funktionen.
\begin{equation}
  f_{Typer}(T_1 T_2 ... : Typer) \rightarrow f_{Termer}((t_1 : T_1) (t_2 : T_2) ...)
\end{equation}

Som de flesta funktionella programmeringspråk så har \coq stöd för ad hoc
polymorphism som även kallas överlagring. Detta innebär att typerna i
funktionen bestämms av sammanhanget användaren behöver då aldrig ge Typerna som
parametrar till den första funktionen utan kompilatorn sköter det. Då \coq är
ett strikt typat språk används typklasser för att beskriva vilka typer som är
tillåtna.

I följande exempel defineras en lista som kan innehålla värden av alla möjliga
typer. Den undre definitionen i exemplet är en rekursiv funktion som lägger
till ett ellement i en lista. Denna funktionen är polymorfisk och fungerar
alltså på alla listor oberoende vilken typ på värden som listan innehåller.
\begin{lstlisting}
Inductive list (X:Type) : Type :=
  | nil  : list X
  | cons : X -> list X -> list X .

Fixpoint append (X:Type) (l : list X) (x : X) : list X :=
  match l with
  | nil       => cons X x (nil X)
  | cons a l' => cons X a (append l' x)
  end.
\end{lstlisting}

Om det inte skulle finnas polymorfi i \coq så skulle det inte gå att skapa en
generell listtyp utan det skulle krävas en specifik listtyp för alla andra
typer. Till exempel om vi skulle lagra heltal i en lista så skulle det finnas
en specifik heltalsliste-typ som det bara gick att lagra heltal i.

\section{Beroendetyper}
I en polymorphisk funktion så kan en parameter- eller retur-typ bero på vilka
typer de tidigare parametrarna har haft. I beroendetypning så går vi ett steg
längre och låter parameter- och retur-typerna bero på värdet av en
tidigare parameter.
Nedan finns två funktionsdefinitioner där beroendetypning används.
I exemplena används en klass som heter Vector som är en lista med fast längd. I
det första exemplet så gör vi om en lista till en vektor och vektorn ska då ha
samma längd som listan. I det andra exemplet har vi en funktion som tar bort
det första ellementet från en vektor och resultatet blir då en vektor med ett
mindre ellement. I \coq notation:
\begin{verbatim}
toVector: (list : List) : Vector (lenght list)

removeFirst (v : Vector n) : Vector (n-1))
\end{verbatim}

Det är inte bara funktionella språk som det finns beroendetypade funktioner.
Ett exempel på beroendetypning som de flesta antagligen är bekanta med är
\texttt{printf} i C. Här bestäms antalet parametrar i funktionen av antalet
\% -tecken i den första strängen och vilken typ det ska vara på dessa
parametrar bestäms av vilken bokstav som står efter \%-tecknet.
\begin{verbatim}
printf("%s is %d years old and %f.1cm long", name, age, lenght)
\end{verbatim}

% http://mattam.org/research/publications/Programming_with_Dependent_Types_in_Coq-PPS-260209.pdf

\section{Grund- och taktikspråk}
\coq består av två olika delspråk. Grundspråket kallas Gallina och liknar till
viss del OCaml. Det är i Gallina som de definitioner och funktioner som ska
bevisas skrivs.

\coq innehåller också ett taktikspråk som heter Ltac och innehåller olika
taktiker för att påverka de hypoteser och mål som ska bevisas. Ltac gör det
möjligt att använda samma metoder i \coq som man använder när man skapar ett
pappersbevis. Då \coq är en interaktiv teorembevisare så när en viss taktik
används uppdateras de hypoteser och mål som ska bevisas och användaren
anger då en ny taktik och detta fortsätter tills alla målen är lösta.

\section{\coq's Historia}
Det är i huvudsak två vetenskaper som ligger till grund för \coq, nämligen
datavetenskap (eng. computer science) och formellbevisföring. Datavetenskapens
vagga startar med en avhandling av Don Knuth, professor på Stanfords
Universitet, år 1968. Det skulle visa sig ta 30 år av forskning efter
avhandligen att fastställa ett rigoröst område. På liknande sätt var även den
rigorösa grunden av formellbevisföring under utveckling under denna tidsepok.
De största genombrotten gjordes dock under 1930-talet av Gentzen, Gödel och
Herbrand. 70 år senare stod \coq som slutprodukt efter en lång serie av projekt.

\section{Tillämpningar}
\coq har, som redan nämnts, använts till att bevisa olika stora satser och även
en del andra programvaror. Det har skett genom stora forskningsprojekt.

\begin{itemize}
\item Fyrfärgssatsen \autocite{gonthier2008formal}\cite{gonthier2005computer}
säger att varje karta kan färgläggas med fyra färger så att inga två
angränsande länder får samma färg. Den visades 1976 med ett bevis som var över
hundra sidor långt. Det som innehöll fallanalys av 10000 fall gjord för hand,
och dessutom en datorprogrammerad fallanalys av miljontals fall. Det gjorde det
mycket svårt att kontrollera om beviset var korrekt. Beviset formaliserades och
verifierades i \coq 1995. Detta arbete ledde till utvecklandet av \ssr.

\item CompCert\autocite{compcert} är ett projekt som utforskar möjligheten att
utveckla formellt bevisade kompilatorer. Anledningen att man vill ha en
formellt bevisad kompilator är att vid vissa optimeringar så kan kompilatorn
skapa buggar och beräkningsfel. Att kompilatorn är formellt bevisad innebär att
det finns ett matematiskt bevis, som kan kontrolleras genom en mekanisk check,
för att den exekverbara koden beter sig så som står föreskrivet i källkoden.
Rent konkret innebär detta att man är garanterad att den exekverbara koden inte
innehåller buggar som är skapade av kompilatorn. Huvudresultatet av detta är en
fungerande C-kompilator som stödjer hela ANSI C (som är den första
standardiserade veritionen av C) med ett få undantag. Vad gäller prestanda så
är CompCert's kompilator snabbare än GCC's (Gnu Compiler Collection)
C-kompilator när denna inte har några optimeringar.

\item Feit-Thompsons eller Odd Order Theorem är en sats inom algebra som säger
att en ändlig grupp alltid är lösbar om dess ordning är udda
\cite{gonthier2013oddorderproof}. Det ursprungliga beviset är över 200 sidor
långt\cite{feit1963}. Storleken på beviset och därmed risken för någon dold
miss i det är en av anledningarna till varför det ansågs intressant att
formalisera i \coq. I samband med formaliseringen av beviset har även en stor
del av matematisk gruppteori formaliserats och verifierats.
\end{itemize}

\section{Alternativ till \coq}
Det finns många olika program och språk för att formalisera och bevisa olika
former av matematik eller logik. En sak som de flesta har gemensamt är att de
är uppbyggd på funktionella programmeringsparadigmer.

\begin{itemize}
\item Agda är utvecklat på Chalmers och påminner till stor del om Haskell. Till
skillnad mot \coq så finns det inga inbyggda taktiker. Agda är inte lika
matematiskt inriktad som \coq utan används mer till att bevisa korrekthet hos
%är vi säkra på det här med matematiskt inriktat?
program. En fördel med Agda är att alla Unicodetecken är tillåtna vilket gör
det enkelt att skriva sina program och bevis på samma sätt som man skulle göra
det på papper.

\item Z3 är ett språk som har utvecklats av Microsoft för att förenkla och
bevisa olika teorem. Kan användas tillsammans med flera stora ickefunktionella
språk som Python, C och .NET.

\item HOL, Högre Ordningens Logik är en av de första interaktiva
teorembevisarna och HOL-light som är en vidareutveckling av det används idag av
Intel för att bevisa att vissa hårdvarukomponenter fungerar korrekt.

\item F* är en vidareutveckling av F\# och används för att verifiera och bevisa
egenskaper hos program. En av de större skillnaderna från F\# är att F* har
stöd för beroendetyper. F* är en del av .NET vilket gör att bevis och kod som
är skriven i F* går att använda i all andra .NET språk. Stora delar av F* är
formaliserade och bevisade i \coq.
\end{itemize}

\begin{comment}
Källor och annat material
HOL http://www.cl.cam.ac.uk/~jrh13/hol-light/
Z3 http://research.microsoft.com/en-us/um/redmond/projects/z3/old/
F* http://research.microsoft.com/en-us/projects/fstar/
\end{comment}
