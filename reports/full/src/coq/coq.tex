\section{Coq}

\subsection{Coq's Historia}
Det är i huvudsak två vetenskaper som ligger till grund för Coq, nämligen
datavetenskap (eng. computer science) och formellbevisföring. Datavetenskapens
vagga startar med en avhandling av Don Knuth, professor på Stanfords
Universitet, år 1968. Det skulle visa sig ta 30 år av forskning efter
avhandligen att fastställa ett rigoröst område. På liknande sätt var även den
rigorösa grunden av formellbevisföring under utveckling under denna tidsepok.
De största genombrotten gjordes dock under 1930-talet av Gentzen, Gödel och
Herbrand. 70 år senare stod Coq som slutprodukt efter en lång serie av projekt.

\subsection{Tillämpningar}
Coq har, som redan nämnts, använts till att bevisa olika stora satser och även
en del andra programvaror. Det har skett genom stora forskningsprojekt.

\begin{itemize}

\item Fyrfärgssatsen\autocite{gonthier2008formal} säger att, givet varje möjlig
uppdelning av ett plan i sammanhängande regioner, så krävs det högst fyra
färger för att färglägga alla regionerna så att inga angränsande regioner har
samma färg. Två regioner anses vara angränsande om de delar en gemensam kant
som inte är ett hörn.

Under årens lopp har det förekommit många olika bevis och motbevis som har
bevisats vara falska och  det var först 1976 som det första korrekta beviset
kom. Stora delar av beviset var då gjorda i en dator och det gick inte att
bekräfta att dessa delar var korrekta. För att bevisa att datordelarna av
beviset var korrekta så formaliserades beviset i Coq och detta ledde till
utvecklandet av SSReflect.

\item CompCert\autocite{compcert} är ett projekt som utforskar möjligheten att
utveckla formellt bevisade kompilatorer. Anledningen att man vill ha en
formellt bevisad kompilator är att vid vissa optimeringar så kan kompilatorn
skapa buggar och beräkningsfel. Att kompilatorn är formellt bevisad innebär att
det finns ett matematiskt bevis, som kan kontrolleras genom en mekanisk check,
för att den exekverbara koden beter sig så som står föreskrivet i källkoden.
Rent konkret innebär detta att man är garanterad att den exekverbara koden inte
innehåller buggar som är skapade av kompilatorn. Huvudresultatet av detta
är en fungerande C-kompilator som stödjer hela ANSI C (som är den första
standardiserade veritionen av C) med ett få undantag.
Vad gäller prestanda så är CompCert's kompilator snabbare än
GCC's (Gnu Compiler Collection) C-kompilator när denna inte
har några optimeringar.

\item Feit-Thompsons är en sats inom matematisk gruppteori som säger att en
ändlig grupp alltid är lösbar om dess ordning är
udda\cite{aschbacher2004status}. Denna sats bevisades av Walter Feit och John
Griggs Thompson 1963. Beviset för Feit-Thompsons sats är stort och sträcker sig
över två volymer, det är mycket material för en person att sätta sig in i och
verifiera för hand. Storleken på beviset och därmed möjligheten till någon dold
miss i beviset är en av anledningarna till varför beviset för Feit-Thompsons
sats är intressant att formalisera i Coq. I samband med formaliseringen av
beviset så har även en stor del av matematisk gruppteori verifierats.

\end{itemize}

\subsection{Alternativ till Coq}
Det finns många olika program och språk för att formalisera och bevisa olika
former av matematik eller logik. En sak som de flesta har gemensamt är att de
är uppbyggd på funktionella programmerings paradigmer.

\begin{itemize}
\item Agda är utvecklat på Chalmers och påminner till stor del om Haskell. Till
skillnad mot Coq så finns det inga inbyggda taktiker.  Agda är inte lika
matematiskt inriktad som Coq utan används mer till att bevisa korrekthet hos
program. En fördel med Agda är att alla Unicodetecken är tillåtna vilket gör
det enkelt att skriva sina program och bevis på samma sätt som man skulle göra
det på papper.

\item Z3 är ett språk som har utvecklats av Microsoft för att förenkla och bevisa
olika teorem. Kan användas tillsammans med flera stora ickefunktionella språk
som Python, C och .NET.

\item HOL, Högre Ordningens Logik är en av de första interaktiva teorembevisarna och
HOL-light som är en vidareutveckling av det används idag av Intel för att
bevisa att vissa hårdvarukomponenter fungerar korrekt.

\item F* är en vidareutveckling av F\# och används för att verifiera och bevisa
egenskaper hos program. En av de större skillnaderna från F\# är att F* har
stöd för beroendetyper. F* är en del av .NET vilket gör att bevis och kod som
är skriven i F* går att använda i all andra .NET språk. Stora delar av F* är
formaliserade och bevisade i Coq.
\end{itemize}

\begin{comment}
Källor och annat material
HOL http://www.cl.cam.ac.uk/~jrh13/hol-light/
Z3 http://research.microsoft.com/en-us/um/redmond/projects/z3/old/
F* http://research.microsoft.com/en-us/projects/fstar/
\end{comment}

\subsection{Curry-Howard isomorphism}
Enligt Curry Howard isomorpismen så är propositioner samma sak som typer och
bevis är samma sak som program. Om vi tar en närmare titt på
funktionsdefinitionen $a \rightarrow b$ så kan vi tolka det som att givet ett
bevis för a så får vi ett bevis för b.
\begin{align*}
  Propotioner &= Typer \\
  Bevis       &= Programs
\end{align*}

\subsection{Polymorphism}
En kort och enkel förklaring till polymorphism är att en funktion kan appliceras
på flera olika typer av parametrar.
En korrektare förklaring till polymorphism är att en polymorpisk funktion
består av två olika delfunktioner. Den första av dessa funktioner tar
typer som parametrar och returnerar den andra delfunktionen vars parametertyper
beror på parametrarna i den första funktionen.
\begin{equation}
f_{Typer}(T_1 T_2 ... : Typer)
\rightarrow f_{Termer}((t_1 : T_1) (t_2 : T_2) ...)
\end{equation}

Som de flesta funktionella programmeringspråk så har Coq stöd för ad hoc
polymorphism som även kallas överlagring. Detta innebär att typerna i
funktionen bestämms av sammanhanget användaren behöver då aldrig ge
Typerna som parametrar till den första funktionen utan kompilatorn sköter
det.
Då coq är ett strikt typat språk används
typklasser för att beskriva vilka typer som är tillåtna.

I följande exempel defineras en lista som kan innehålla värden av alla möjliga
typer. Den undre definitionen i exemplet är en rekursiv funktion som lägger till
ett ellement i en lista. Denna funktionen är polymorphisk och fungerar alltså på
alla listor oberoende vilken typ på värden som listan innehåller.
\begin{lstlisting}
Inductive list (X:Type) : Type :=
  | nil  : list X
  | cons : X -> list X -> list X .

Fixpoint append (X:Type) (l : list X) (x : X) : list X :=
  match l with
  | nil       => cons X x (nil X)
  | cons a l' => cons X a (append l' x)
  end.
\end{lstlisting}

Om det inte skulle finnas polymorphism i \coq så skulle det inte
gå att skapa en generell listtyp utan det skulle krävas en specifik
listtyp för alla andra typer. Till exempel om vi skulle lagra heltal
i en lista så skulle det finnas en specifik heltalsliste-typ som det bara
gick att lagra heltal i.

\subsection{Beroendetyper}
I en polymorphisk funktion så kan en parameter- eller retur-typ bero på vilka
typer de tidigare parametrarna har haft. I beroendetypning så går vi ett steg
längre och låter parameter- och retur-typerna bero på värdet av en
tidigare parameter.
Nedan finns två funktionsdefinitioner där beroendetypning används.
I exemplena används en klass som heter Vector som är en lista med fast längd. I
det första exemplet så gör vi om en lista till en vektor och vektorn ska då ha
samma längd som listan. I det andra exemplet har vi en funktion som tar bort
det första ellementet från en vektor och resultatet blir då en vektor med ett
mindre ellement. I \coq notation:
\begin{verbatim}
toVector: (list : List) : Vector (lenght list)

removeFirst (v : Vector n) : Vector (n-1))
\end{verbatim}

Det är inte bara funktionella språk som det finns beroendetypade funktioner.
Ett exempel på beroendetypning som de flesta antagligen är bekanta med är
\texttt{printf} i C. Här bestäms antalet parametrar i funktionen av antalet
\% -tecken i den första strängen och vilken typ det ska vara på dessa
parametrar bestäms av vilken bokstav som står efter \%-tecknet.
\begin{verbatim}
printf("%s is %d years old and %f.1cm long", name, age, lenght)
\end{verbatim}

% http://mattam.org/research/publications/Programming_with_Dependent_Types_in_Coq-PPS-260209.pdf

\subsection{Grund och taktikspråk}
Coq består av två olika delspråk. Grundspråket kallas Gallina och liknar till
viss del OCaml. Det är i Gallina som de definitioner och funktioner som ska
bevisas skrivs.

Coq innehåller också ett taktikspråk som heter Ltac och innehåller olika
taktiker för att påverka de hypoteser och mål som ska bevisas. Ltac gör det
möjligt att använda samma metoder i Coq som man använder när man skapar ett
pappersbevis. Då \coq är en interaktiv teorembevisare så när en viss taktik
används uppdateras de hypoteser och mål som ska bevisas och användaren
anger då en ny taktik och detta fortsätter tills alla målen är lösta.
