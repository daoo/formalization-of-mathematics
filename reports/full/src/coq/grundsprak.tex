\section{Grundspråk}
Grundspråket i \coq{} kallas \textsc{Gallina} och är som vi redan nämnt
funktionellt och beroendetypat. Liksom i andra programmeringsspråk definierar
man funktioner och deras typ. Det är dock inte tillåtet i \coq{} att definiera
funktioner som inte terminerar. Det vill säga, i de flesta språk kan man skriva
\begin{verbatim}
loop :: Integer -> Integer
loop n = loop (n + 1)
\end{verbatim}
och kompilatorn kommer att kompilera och låta dig köra funktionen. Men
\verb+loop+ kommer aldrig ge något resultat och kommer snurra i all evighet.
Kompilatorn till \coq{} tillåter inte denna typen av definitioner utan kommer att
ge kompileringsfel om den stöter på liknande funktioner. För att \coq{} ska
acceptera en rekursiv funktion krävs det att någon av parametrarna i det
rekursiva anropet närmar sig basfallet eller så måste användaren ange ett bevis
för att funktionen är garanterad att avsluta. Det går till exempel inte att
implementera Collatz funktion som beskrivs i ekvation~\ref{eq:collatz} även om
den terminerar för alla hittills kända värden.

\begin{equation}
\label{eq:collatz}
T(n) = \left\{\begin{matrix} n/2, & \mbox{om }n\equiv0\mbox{ (mod 2)} \\ 3n+1,
                         & \mbox{om }n\equiv1\mbox{ (mod 2)} \end{matrix}\right.
\end{equation}

För att göra en definition i \coq{} använder man nyckelordet \C{Definition}. En
definition av funktionskomposition kan se ut så här:
\begin{lstlisting}
Definition compose (a b c: Type) (f: b -> c) (g: a -> b) (x: a) : c := f (g x).
\end{lstlisting}
I definitionen av \C{compose} måste vi först ge typerna \C{a}, \C{b} och \C{c}
explicit, till skillnad från i \haskell{} där de är definierade implicit.
Parametrarna \C{f}, \C{g} och \C{x} följer sedan direkt. I \coq{} används punkt
(\C{.}) för att avsluta en definition likt hur man avslutar ett påstående
(\emph{engelska: statement}) i C med semikolon (;). För att anropa \C{compose}
måste man ge typer, funktioner och värdet. Till exempel:
\begin{lstlisting}
Definition add1 (x: nat) := x + 1.
Definition mul2 (x: nat) := x * 2.

Definition example : nat := compose nat nat nat add1 mul2 5.
\end{lstlisting}
Notera att programmerare är lata och vill gärna skriva så lite som möjligt. I
\coq{} kan man därför se till att typ parametrarna \C{a}, \C{b} och \C{c}
automatiskt sätts in på rätt ställe i funktionsanropen. För att göra detta
behöver man dock ändra på definitionen av \C{compose} och sätta
klammerparenteser runt parametrarna istället:
\begin{lstlisting}
Definition compose' {a b c: Type} (f: b -> c) (g: a -> b) (x: a) : c := f (g x).
\end{lstlisting}
Vi kan även använda så kallade lambdauttryck, vilket betyder funktioner
definierade utan namn som används direkt där man skriver dem. Syntaxen för
lambdauttryck är \C{fun parameters => application}. Med dessa förändringar kan
vi skriva
\begin{lstlisting}
Definition example' : nat := compose' (fun x => x + 1) (fun x => x * 2) 5.
\end{lstlisting}
\coq{} har även datatyper och de definieras med hjälp av \C{Inductive}. En
definition av naturliga tal kan se ut så här:
\begin{lstlisting}
Inductive nat : Type :=
  | O: nat
  | S: nat -> nat.
\end{lstlisting}
Typen för varje konstruktor måste ges explicit. Vi kan även definiera vektorer, för
att blanda in lite beroendetypning:
\begin{lstlisting}
Inductive vec (X: Type) : nat -> Type :=
  | empty: vec X O
  | element: forall n, X -> vec X n -> vec X (S n).
\end{lstlisting}
% Notera att jag har fuskat här och inte tagit med Implicit Argument raderna
% som behövs för att man inte ska behöva ange typerna hela tiden.
Sedan kan vi definiera en typsäker version av \C{tail} med hjälp av
mönstermatchning:
\begin{lstlisting}
Definition tail (X: Type) (n: nat) (v: vec X (S n)) : vec X n :=
  match v with
  | element X _ xs => xs
  end.
\end{lstlisting}
Mönstermatchning innebär att man med hjälp av något värde tillhörande någon typ
bestämmer vad som ska göras för varje konstruktor till typen. Notera i det här
fallet att funktionen bara är definierad för vektorer \C{vec X n} där \C{n >
0}. Vi behöver således inte hantera tomma vektorer i mönstermatchningen. Värt
att notera är likheten mellan mönstermatchningen i \coq{} och \C{case} sats i
\textsc{Haskell}.

För att definiera en rekursiv funktion använder vi \C{Fixpoint} istället för
\C{Definition}:
\begin{lstlisting}
Fixpoint concatenate (X: Type) {n m: nat} (a: vec X n) (b: vec X m) : vec X (n+m) :=
  match a with
  | empty => b
  | element _ x xs => element x (concatenate X xs b)
  end.
\end{lstlisting}

Detta var en mycket kort introduktion till \coq{} som funktionellt
programmeringsspråk och har därför utelämnat bland annat modulhantering,
typklasser, kanoniska strukturer, poster, partiell applicering och notationer.
Dessa ämnen är också viktiga för att förstå \coq{} men är bortom målet med denna
rapport.
