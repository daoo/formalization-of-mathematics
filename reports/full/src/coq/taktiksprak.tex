\section{Taktikspråk}
Taktikspråk i \coq{} heter \textsc{Ltac} och är vad som används för att bygga
de formella bevisen. Kort sagt skrivs bevis genom att först sätta upp satsen
som ska bevisas som en likhet. Sedan använder så kallade taktiker för att
modifierar uttrycket tills uttrycket är lika. Arbetsgången liknar på så sätt
bevis med papper och penna dock mycket mer detaljerat.

För att påbörja ett bevis skrivs först \C{Lemma} eller \C{Theorem} följt av
namnet på beviset. Sedan skrivs kolon (:) följt av satsens kvantifikationer,
det vill säga de objekt uttrycket beror på. Efter det följer ett kommatecken
(,) och satsens påstående följt av punkt (.). Sedan är det tillåtet att skriva
nyckelordet \C{Proof}, men det är bara syntaktiskt socker och betyder inget för
kompilatorn. Till sist kommer beviset självt uppbyggt med hjälp av
\emph{taktiker} följt av \C{Qed.} för att avsluta beviset. För att bevisa till
exempel en tautologi.

\begin{lstlisting}
Theorem tautology : forall (A B C: Prop),
  (A -> B) -> (B -> C) -> A -> C.
Proof.
  intros A B C Hab Hbc Ha.
  apply Hbc.
  apply Hab.
  apply Ha.
Qed.
\end{lstlisting}

Innuti själva beviset finns ett så kallat kontext och en lista med mål som
behöver bevisas. Med kontext menas de objekt som användaren kan använda sig av
i beviset. Genom att använda sig av taktiker modifieras målen och kontextet.
Efter \C{Proof.} i \C{tautology} ser kontexten och målet ut på följande sätt
\begin{lstlisting}
1 subgoals
______________________________________(1/1)
forall A B C : Prop, (A -> B) -> (B -> C) -> A -> C
\end{lstlisting}

Taktiken \C{intros} flyttar antaganden från målet till kontexten så att de kan
användas i andra taktiker. I \C{tautology} flyttar vi upp alla implikationerna
och ger dem namn, det leder till följande kontext och mål
\begin{lstlisting}
1 subgoals
A : Prop
B : Prop
C : Prop
Hab : A -> B
Hbc : B -> C
Ha : A
______________________________________(1/1)
C
\end{lstlisting}

Nu är målet att bevisa \C{C}, det kan vi göra genom att applicera hypotesen
\C{Hbc} med \C{apply} taktiken. Det resulterar i att vi behöver bevisa \C{B}
vilket vi kan göra med hypotesen \C{Hab}. Till sist får vi målet \C{A} vilket
kan bevisas med hypotesen \C{Ha}. Kontexten visar då texten \C{No more
subgoals} och beviset kan avslutas med \C{Qed}.
