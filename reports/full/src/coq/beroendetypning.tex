\section{Beroendetypning}
För att förklara beroendetypning använder vi begreppen \emph{termer} och
\emph{typer}. Förenklat sett kan man säga att termer är värden eller funktioner
och typer är samlingar (matematisk mängder) av värden eller termer. Notera att
varje term har en typ. Vi går nu gå igenom hur ett funktionellt
programmeringsspråk med beroendetypning kan vara uppbyggt.

\subsection{Termer som beror på termer}
Den grundläggande idén med funktionell programmering är att \emph{termer kan
bero på termer}. I ett teoretiskt funktionellt språk skulle det kunna innebära
att givet någon typ, låt oss använda de naturliga talen $\N$ i det här
exemplet, skulle man kunna skriva
\begin{align*}
  &compose : (\N \to \N) \to (\N \to \N) \to \N \to \N \\
  &compose\ f\ g\ x = f\ (g\ x)
\end{align*}
där termerna $f$, $g$ och $compose$ är funktioner och termen $x$ är ett värde.
Här säger man att termen $compose$ beror på termerna $f$, $g$ och $x$.
Funktionen $compose$ definierar funktionskomposition, oftast noterad med en
cirkel som $(f \circ g)\ (x)$.

% TODO: Explain types in more details, with example definition.

\subsection{Termer som beror på typer}
Utöver termer som beror på termer kan man även lägga till \emph{termer som
beror på typer}, vilket brukar kallas polymorfism. För att bygga vidare på vårt
ovan exempel skulle det kunna se ut så här:
\begin{align*}
  &compose' : (b \to c) \to (a \to b) \to a \to c \\
  &compose'\ f\ g\ x = f\ (g\ x)
\end{align*}
Här har vi i stället för att använda typen $\N$ använt de godtyckliga typerna
$a$, $b$ och $c$ vilket innebär att termen $compose'$ beror på typerna $a$, $b$
och $c$. Det låter oss med en enda definition av $compose'$ anropa funktionen
med termer av olika typer, i stället för att definiera $compose$ för varje
kombination av typer.

Ytterligare ett exempel på polymorfism är \emph{polymorfiska typer}. Till
exempel kan vi skriva
\begin{align*}
  &\boldsymbol{data}\ List\ a\ \boldsymbol{where} \\
  &\ \ Nil : List\ a \\
  &\ \ Cons : a \to List\ a \to List\ a
\end{align*}
där $List\ a$ är en polymorfisk datatyp för listor. Vi skulle sedan kunna ge
typen för en funktion som ger tillbaka det första elementet ur en lista:
\begin{align*}
  head : List\ a \to a
\end{align*}

\subsection{Typer som beror på typer}
Nästa steg i utvecklingen mot beroendetypning är att låta \emph{typer bero på
typer}. Med denna funktionalitet kan man definiera så kallade typoperatorer.
Typoperatorer är funktioner som tar in typer som parametrar och ger ut typer
som resultat, till skillnad från vanliga funktioner som tar in termer som
parametrar och ger ut termer som resultat.

Ett exempel på en typoperator är den kartesiska produkten ($\times$).
Typoperatorn $\times$ tar in två typer och ger ut en ny typ ofta kallad en
produkttyp. Eftersom typer är mängder blir kartesiska produkten av två typer
alla möjliga kombinationer av typernas element. Vi skulle även kunna använda
oss av en typoperator som representerar union av disjunkta mängder ($+$).
Unionen av två typer skulle ge oss en ny typ, ofta kallad en summatyp, som
innehåller alla element från båda typerna.

\subsection{Typer som beror på termer}
Det sista steget i byggandet är att lägga till \emph{typer som beror på
termer}, även känt som beroendetypning. Beroendetypning innebär väldigt
förenklat att man på typnivå kan använda samma funktioner och värden som man
kan använda på termnivå. Till exempel låter detta oss definiera vektorer, det
vill säga listor av en bestämd längd. Detta gör vi genom att lägga till en
parameter som representerar vektorns längd i typen och öka denna med ett när vi
lägger till ett element.
\begin{align*}
  &\boldsymbol{data}\ Vector\ a\ (n \in \N)\ \boldsymbol{where} \\
  &\ \ Empty : Vector\ a\ 0 \\
  &\ \ Element : a \to (m \in \N) \to Vector\ a\ m \to Vector\ a\ (m+1)
\end{align*}
Sedan kan vi ge typen till funktioner som till exempel att slå ihop två
vektorer:
\begin{align*}
  concatenate : n \in \N \to m \in \N \to Vector\ a\ n \to Vector\ a\ m \to Vector\ a\ (n+m)
\end{align*}
eller en funktion som tar ut första elementet på samma sätt som $head$ för
listor:
\begin{align*}
  head' : (n \in \N) \to Vector\ a\ (n+1) \to Vector\ a\ n
\end{align*}
Notera att denna funktionen inte tillåter tomma vektorer som argument eftersom
man inte kan ta bort ett element som inte finns, kompilatorn kontrollerar detta
genom att matcha typen på den inskickade vektorn med värdet $n+1$. Om vektorn
har en typ där värdet $n = 0$ matchar inte typen eftersom det inte finns $m \in
\N$ sådant att $m+1 = 0$. Detta till skillnad från definitionen av $head$ för
listor som helt enkelt inte kan definieras för tomma listor och i \haskell{}
kraschar program som använder $head$ på tomma listor.

Värt att notera är att inte bara funktionella språk har beroendetypade
funktioner. Ett exempel på beroendetypning som de flesta inom datavetenskap är
bekanta med är \texttt{printf} i \textsc{C}. Här bestäms antalet parametrar i
funktionen av antalet \% -tecken i den första strängen och vilken typ det ska
vara på dessa parametrar bestäms av vilken bokstav som står efter respektive
\%-tecken.
\begin{verbatim}
printf("%s is %d years old and %f.1cm long", name, age, length)
\end{verbatim}

% TODO: Totalitet
