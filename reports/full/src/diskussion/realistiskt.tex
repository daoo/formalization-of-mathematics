\section{Är det realistiskt att formalisera bevis och kod?}
Implementationen av algoritmen och beviset i \coq{} var mycket tidskrävande,
även för en relativt liten algoritm med ett kortare matematiskt bevis. Flera
veckors arbete med flera personer behövdes för att färdigställa beviset.
Jämförelsevis tog implementationen i \haskell{} samt testning med QuickCheck
mindre än en dag. Frågan är om detta är en rättvis jämförelse eftersom vi hade
mycket mer erfarenhet av \haskell{} än \coq{}. Även om mer erfarenhet skulle ha
kortat ner utvecklingstiden markant tar formella bevis flera storleksordningar
längre tid att utveckla än att implementera algoritmen i till exempel \haskell{}.

Erfarenhet är inte den enda lösningen, \coq{} är ett forskningsprojekt och
lider av diverse problem. Till exempel interaktionsproblem med programvaran, de
grafiska gränssnitten som finns har en del buggar och saknar många funktioner
som finns i andra moderna utvecklingsmiljöer för andra språk. Vidare är
dokumentationen för standardbiblioteket dålig eller obefintlig. När man sitter
och bevisar går väldigt mycket tid åt till att enbart sitta och söka i
källkoden för att hitta rätt lemma. Om man kan lösa dessa problemen skulle
tiden som behövs kortas ner avsevärt.

Att använda \coq{} för att bevisa kod är kanske i dagsläget bara relevant om
det är väldigt viktigt att koden är korrekt. För matematiker kan det vara ett
mer relevant verktyg eftersom \coq{} stoppar direkt när man gjort fel istället
för att små enkla fel smyger sig igenom hela beviset.
