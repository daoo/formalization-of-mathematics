\subsection{Implementeringen av Toom-Cook}
Den version av Toom-Cook som beskrevs i avsnitt~\ref{in:definition} och som
implementerades i avsnitt~\ref{section:formalg} är generell och på abstrakt
vilket är bra då den ska bevisas. Med vår implementation kan inga beräkningar
utföras då typen av polynom i \ssr är låst, se avsnitt~\ref{avsnitt:ssr}. Om
det hade funnits tid skulle nästa steg i projektet varit att implementera en
eller flera exekverbara versioner av algoritmen och visa att dessa gav samma
resultat vid beräkningar som den abstrakta algoritmen. Det skulle då innebära
att de gav korrekta resultat, så som diskuteras i avsnitt~\ref{disk:formalg}.

Nedan diskuteras några abstrakta algoritmer, dess fördelar och nackdelar, och
andra möjliga val av implementeringar. Några möjliga effektiviseringar som
skulle kunna göras vid implementaion av exekverbara algoritmer tas också upp.

Den abstrakta algoritmen som har implementerats använder sig av
matrismultiplikation och -inverison i evaluerings- och interpolationsstegen.
Det motiverades delvis av praktiska skäl för att de redan är kända
matematiska-operationer (som alla i gruppen redan är bekanta med) och för att
det finns många algoritmer för att optimera matrisoperationer som skulle kunna
användas när en exekverbar algoritm baserad på dem ska implementeras. Detta är
inte det enda sättet att implementera dessa steg på, och den här veriosnen
fungerar inte heller för alla integritetsområden.

% velat få en någorlunda enkel algoritm, då de teoretiska grejerna kring
% polynominterpolation och -evaluering gått bortom projektets ***vad?ramar, gränser, syfte****.

% Det är ett smidigt sätt att representera
% dessa steg, men det fungerar inte för alla integritetsområden
% och det finns alternativa sätt att implementera detta.

I evalueringssteget~\ref{in:evaluation} i algoritmen multipliceras
evalueringsmatrisen $V_e$ med vektorn av koefficienter till polynomet som ska
evalueras och detta är inte det enda möjliga sättet att evaluera polynomen på.
För att optimera en beräkningsbarversion av algoritmen finns möjligheten att
välja en ordning på operationerna i evalueringen som kräver färre operationer
än direkt matrismultiplikation{bodrato2007towards}.

I interpolationssteget~\ref{in:interpol} antas det att interpolationsmatrisen
$V_I$ är inverterbar. Det gör att koefficienterna i polynomet kan fås genom att
multiplicera inversen av $V_I$ med vektorn av polynomets värden i
interpolationspunkterna.

I en kropp (där alla element utom 0 har en multiplikativ invers) kommer $V_I$
vara inverterbar om och endast om ekvationssystemet som ges av \ref{eq:NAME4}
är entydigt lösbart, vilket det kommer vara om interpolationspunkterna
$\alpha_0, \dots, \alpha_d$ är $d + 1$ skilda punkter.

Detta gäller dock inte generellt i ett integritetsområde. Då kan
ekvationssystemet \ref{eq:NAME4} vara entydigt lösbart utan att $V_I$ är
inverterbar, eftersom matris över ett integritetsområde är inverterbar om och
endast om dess determinant är ett inverterbart
element\cite{sombatboriboon2011some}. ($V_I$ kommer dock vara inverterbar i
bråkkroppen över integritetsområdet.)

Därför i en del integritetsområden, där det finns för få inverterbara element
eller där de element som ändå ger upphov till en inverterbar matris är inte
lämpade för en effektiv matrisinversion, kan det vara lämpligt eller nödvändigt
använda andra ekvationslösningsmetoder än matrisinversion.

% För att det alls ska vara möjligt att hitta en
% inverterbar interpolationsmatris $V_I$ måste det finnas $\alpha_0, \dots, \alpha_d$ i
% integritetsområdet så att
% $\det V_I = \prod_{0 \leq i < j \leq d} (\alpha_i - \alpha_j)$ är inverterbar. Även om
% det skulle finnas det kanske de inte är de mest lämpliga interpolationspunkterna i
% andra avseenden. Ska metoder som bygger på matrisinversion användas i interpolationssteget

I de integritetsområden där det fungerar finns också möjligheter att
effektivisera ekvationslösningen genom att använda till exempel
$LU$-faktorisering. Bodrato\cite{bodrato2007towards}\cite{bodrato2007integer}
beskriver en algoritm för att givet interpolationspunkterna söka efter optimala
följder av evaluerings- och interpolationsoperationer.

En generell men exekverbar version av Toom-Cook skulle kanske kunna
implementeras. Den skulle dock kräva en av två saker. Antingen att de önskade
interpolationspunkterna gavs som argument vid varje applikation av algoritmen
eller att interpolationspunkter för varje möjligt integritetsområde sparades
eller genererades med hjälp av någon algoritm när Toom-Cook anropades. Eftersom
$\toom$ går att definera för alla heltal större än 0 skulle en helt generell
implementation kräva att det gick att generera godtyckligt många lämpliga
interpolationspunkter. Det går natutligtvis inte i ett ändligt
integritetsområde.

Vid praktiska tillämpningar är det dock mer rimligt att implementera en eller
några instanser av $\toom$, till exempel Toom-Cook-3 och Toom-Cook-5, för något
eller några typer av
%****familjer(jag vill säga att de ska ha en gemensam isomorf delring eller nåt
%typ)*** av
integritetsområden.
Bodrato\cite{bodrato2007a}\cite{bodrato2007towards}\cite{bodrato2007integer}
beskriver optimerade implementationer för karaktärestik 2, 3, 5 och 0 av bland
annat Toom-Cook-3. GNU Multiple Precision Arithmetic Library använder
Toom-Cook-3, 4, 6.5 och 8.5 (där 6.5 och 8.5 är algoritmer speciellt anpassade
för att multiplicera operander med stor storleksskillnad) för att multiplicera
heltal inom olika storleksgränser.

Då kan algoritmen appliceras utan att man behöver ge interpolationspunkter som
argument och man behöver inte räkna ut evaluerings- och
interpolationsmatriserna vid varje applikation.

Den faktiska implementationen av ett steg i algoritmen kan ha en stor betydelse
för hur komplext det sedan är att visa algoritmens korrekthet. Till exempel
$recompose\_split$ är ett lemma i vårat formella bevis som säger att våran
uppdelningsfunktion, ``split'', är högerinvers till sammansättningsfunktionen,
``recompose''. Trots att det här är ett synnerligen enkelt lemma så är det
formella beviset relativt långt och komplicerat. Detta beror till stor del på
att vi har implementerat uppdelningen, ``split'', med polynomdivision och modulo
för polynom vilket i sin tur ledde till svårigheter då vi behövde arbeta med de
listor som utgör den underliggande strukturen för polynomtypen. Det bör finnas
en mer naturlig definition av uppdelningsfunktionen som gör att beviset för
$recompose\_split$ går att skriva på ett fåtal rader.

Vi diskuterade en alternativ implementation av split definierad utan
polynomdivision och modulo för polynom som såg ut på följande vis:
\begin{lstlisting}
Definition split (n b: nat) p : {poly {poly R}} :=
  \poly_(i < n) \poly_(j < b) p`_(i*b+j).
\end{lstlisting}
Vi lyckades dock inte hitta ett formellt bevis med hjälp av denna definition.

