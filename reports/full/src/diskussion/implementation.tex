\section{Implementeringen av Toom-Cook}
% TODO: Dela upp denna sektionen i flera delar?
% Den version av Toom-Cook som beskrevs i avsnitt~\ref{sec:definition} och som
% implementerades i avsnitt~\ref{sec:formell} är generell.
% vilket är bra då den ska bevisas.
% I vår implementation kan dock inga beräkningar
% utföras då typen av polynom i \ssr{} är låst, se avsnitt~\ref{avsnitt:ssr}. Om
% det hade funnits tid skulle nästa steg i projektet varit att implementera en
% eller flera exekverbara versioner av algoritmen och visa att dessa gav samma
% resultat vid beräkningar som den abstrakta algoritmen. Det skulle då innebära
% att de gav korrekta resultat, så som diskuteras i avsnitt~\ref{disk:formalg}.

Nedan diskuteras några val för definitionen och implementationen av algoritmen i
avsnitt~\ref{sec:definition} och avsnitt~\ref{sec:formell}. Några möjliga
effektiviseringar vid implementation av exekverbara algoritmer tas också upp.

Den version av algoritmen som implementerats i det här projektet använder sig av
matrismultiplikation och -invertering i evaluerings- och interpolationsstegen.
Det motiverades delvis av praktiska skäl: det skulle gått utanför projektets
ramar att i detalj undersöka alternativa metoder för dessa steg.
% för att de redan är kända
% matematiska-operationer (som alla i gruppen redan är bekanta med)
Dessutom finns det många algoritmer för att optimera matrisoperationer som skulle kunna
användas för att effektivisera en exekverbar Toom-Cook-algoritm baserad på den i det här
projektet. Det finns dock inte andra möjliga sätt att implementera interpolations- och
evalueringsstegen på.

%det fungerar inte heller för alla integritetsområden.

% velat få en någorlunda enkel algoritm, då de teoretiska grejerna kring
% polynominterpolation och -evaluering gått bortom projektets ***vad?ramar, gränser, syfte****.

% Det är ett smidigt sätt att representera
% dessa steg, men det fungerar inte för alla integritetsområden
% och det finns alternativa sätt att implementera detta.

\paragraph{Evaluering}
I avsnitt~\ref{sec:evaluering} i algoritmen multipliceras
evalueringsmatrisen $V_e$ med vektorn av koefficienter till polynomet som ska
evalueras. Detta är inte det enda möjliga sättet att evaluera polynomen på.
För att optimera en beräkningsbar version av algoritmen kan det finnas möjlighet att
välja ordning på additionerna och multiplikationerna i evalueringen så att det
krävs färre operationer än vad som skulle ske med matrismultiplikation enligt
definitionen\cite{bodrato2007towards}.

\paragraph{Interpolation}
I avsnitt~\ref{sec:interpol} antas det att interpolationsmatrisen
$V_I$ är inverterbar. Det gör att koefficienterna i polynomet kan fås genom att
multiplicera inversen av $V_I$ med vektorn av polynomets värden i
interpolationspunkterna.

I en kropp (där alla element utom 0 har en multiplikativ invers) kommer $V_I$
vara inverterbar om och endast om ekvationssystemet som ges av ekvation(\ref{eq:NAME4})
är entydigt lösbart, vilket det kommer vara om interpolationspunkterna
$\alpha_0, \dots, \alpha_d$ är $d + 1$ skilda punkter.

Detta gäller dock inte generellt i ett integritetsområde. Då kan
ekvationssystemet(\ref{eq:NAME4}) vara entydigt lösbart utan att $V_I$ är
inverterbar, eftersom en matris över ett integritetsområde är inverterbar om
och endast om dess determinant är ett inverterbart
element\cite{sombatboriboon2011some}. ($V_I$ kommer dock vara inverterbar i
bråkkroppen över integritetsområdet.)

Därför kan det i en del integritetsområden, där det finns för få inverterbara element
eller där de element som ändå ger upphov till en inverterbar matris inte är
lämpade för en effektiv matrisinvertering, vara lämpligt eller nödvändigt att
använda andra ekvationslösningsmetoder än matrisinvertering.

% För att det alls ska vara möjligt att hitta en
% inverterbar interpolationsmatris $V_I$ måste det finnas $\alpha_0, \dots, \alpha_d$ i
% integritetsområdet så att
% $\det V_I = \prod_{0 \leq i < j \leq d} (\alpha_i - \alpha_j)$ är inverterbar. Även om
% det skulle finnas det kanske de inte är de mest lämpliga interpolationspunkterna i
% andra avseenden. Ska metoder som bygger på matrisinversion användas i interpolationssteget

I de integritetsområden där matrisinverteringen fungerar finns också
möjligheter att effektivisera ekvationslösningen genom att använda till exempel
$LU$-faktorisering. Bodrato\cite{bodrato2007towards}\cite{bodrato2007integer}
beskriver en algoritm för att givet interpolationspunkterna söka efter optimala
följder av evaluerings- och interpolationsoperationer.

\paragraph{Generell eller specifik implementation}
En generell men exekverbar version av Toom-Cook skulle kanske kunna
implementeras. Den skulle dock kräva en av två saker. Antingen att de önskade
interpolationspunkterna gavs som argument vid varje applikation av algoritmen
eller att interpolationspunkter för varje möjligt integritetsområde sparades
eller genererades med hjälp av någon algoritm när Toom-Cook anropades. Eftersom
Toom-Cook-\emph{m} går att definera för godtyckligt stora heltal $m$ skulle en
helt generell implementation kräva att det gick att generera godtyckligt många
lämpliga interpolationspunkter eftersom antalet interpolationspunkter är $2m-1$.
Det går natutligtvis inte i ett ändligt integritetsområde.

Vid praktiska tillämpningar av Toom-Cook är det dock mer rimligt att implementera
en eller några instanser av den, till exempel Toom-Cook-3 och Toom-Cook-5, för
någon eller några typer av
%****familjer(jag vill säga att de ska ha en gemensam isomorf delring eller nåt
%typ)*** av
integritetsområden.
Bodrato\cite{bodrato2007notes}\cite{bodrato2007towards}\cite{bodrato2007integer}
beskriver optimerade implementationer för karaktäristik 2, 3, 5 och 0 av bland
annat Toom-Cook-3. GNU Multiple Precision Arithmetic Library använder
Toom-Cook-3, -4, -6.5 och -8.5 (där -6.5 och -8.5 är algoritmer speciellt anpassade
för att multiplicera operander med stor storleksskillnad) för att multiplicera
heltal inom olika storleksgränser\cite{gmpdoc}.

När en sådan implementering är gjord kan algoritmen appliceras utan att man
behöver ge interpolationspunkter som argument och man behöver inte räkna ut evaluerings- och
interpolationsmatriserna vid varje applikation.

\paragraph{Implementationens påverkan på beviset} Hur ett steg i algoritmen
implementeras har en stor betydelse för hur komplicerat det sedan är att visa
algoritmens korrekthet. Till exempel $recompose\_split$ är ett lemma i vårt
formella bevis som säger att vår uppdelningsfunktion \C{split} är högerinvers
till sammansättningsfunktionen \C{recompose} under vissa förutsättningar. Trots
att det är ett ganska enkelt lemma så är det formella beviset relativt långt
och komplicerat. Detta beror till stor del på att vi har implementerat
uppdelningen, \C{split}, med polynomdivision och modulo för polynom vilket i
sin tur ledde till svårigheter då vi behövde arbeta med de listor som utgör den
underliggande strukturen för polynomtypen. Det bör finnas en mer naturlig
definition av uppdelningsfunktionen som gör att beviset för $recompose\_split$
går att skriva på ett fåtal rader.

Vi diskuterade en alternativ implementation av \C{split} definierad utan
polynomdivision och modulo för polynom som såg ut på följande vis:
\begin{lstlisting}
Definition split (n b: nat) p : {poly {poly R}} :=
  \poly_(i < n) \poly_(j < b) p`_(i*b+j).
\end{lstlisting}
Vilket betyder $\sum_{i=0}^{n-1}\sum_{j=0}^{b-1}(p_{ib+j}x^j)y^i$ där
inparametrarna är samma som i den ursprungliga \C{split}-definitionen. Vi hann
dock inte testa om detta var en effektivare definition.
