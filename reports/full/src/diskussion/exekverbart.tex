\section{Formalisering och matematiska algoritmer}
\label{sec:exekverbar}
Mörtberg, Dénès och Siles presenterar i \cite{denes2012refinement} en
metodologi för att implementera effektiva matematiska algoritmer och visa deras
korrekthet. De delar in processen i tre steg: först definieras algoritmen med
hjälp av de låsta men uttrycksfulla datatyperna i \ssr{}:s bibliotek och
bevisas, sedan översätts \ssr{}-algoritmen till en algoritm definierad med
enklare typer i \ssr{} som det går att utföra beräkningar med, och denna visas
ge samma resultat som algoritmen i det första steget för motsvarande argument.
Till sist kan den exekverbara algoritmen översättas till något annat språk,
till exempel \haskell{}.

Implementationen av Toom-Cook i det här projektet kan sägas svara mot första
steget i den här processen. Om tid hade funnits skulle nästa steg varit att
implementera en exekverbar algoritm.

I \ssr{} representeras ett polynom som ett par av en lista och ett bevis för
att det sista elementet i listan inte är 0 och matriser som ändliga funktioner,
se avsnitt~\ref{sec:formellbevis}. I den exekverbara versionen av algoritmen skulle
polynom representeras med vanliga listor och matriser med listor av listor.
Polynom- och matrisoperationer för dessa representationer finns implementerade
i biblioteket \coq{}EAL\cite{coqeal} till \coq{}. Där finns också funktioner
som översätter mellan olika representationer.

Genom att sedan visa att samma resultat fås av att applicera den redan
implementerade Toom-Cook-algoritmen och sedan översätta resultatet till
polynomlistor och att översätta argumenten till listor först och sedan
applicera listversionen av Toom-Cook-algoritmen skulle man särkerställa att
listversionen ger korrekta resultat för de argument som är intressanta, det
vill säga listor där sista elementet är nollskiljt, som därmed är listversioner
av polynom.
