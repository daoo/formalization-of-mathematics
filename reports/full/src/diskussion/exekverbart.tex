\subsection{Formalisering och matematiska algoritmer}
\label{sec:exekverbar}
Mörtberg, Dénès och Siles presenterar i \cite{denes2012refinement} en metodologi för att
implementera effektiva matematiska algoritmer och visa deras korrekthet.
De delar in processen i tre steg: I det första definieras algoritmen med hjälp av de
låsta men uttrycksfulla datatyperna i \ssr{}:s bibliotek och bevisas. I det andra
steget översätts denna algoritm till en algoritm definierad med enklare typer i \ssr{}
som det går att utföra beräkningar med, och denna visar man ger samma resultat som
algoritmen i det första steget för motsvarande argument. I det tredje steget kan
algoritmen från steg 2 översättas till något annat språk, till exempel \haskell{}.

Implementationen av Toom-Cook i det här projektet kan sägas svara mot det
första steget i den här processen. Om tid hade funnits i projektet skulle nästa
steg varit att implementera en exekverbar algoritm som skulle motsvara steg 2 i
processen ovan.

I \ssr{} representeras ett polynom som ett par av en lista och ett bevis för
att det sista elementet i listan inte är 0 och matriser som ändliga funktioner,
se avsnitt~\ref{formbevis}. I den exekverbara versionen av Toom-Cook-algoritmen
skulle polynom representeras med vanliga listor och matriser med listor av
listor. Polynom- och matrisoperationer för dessa representationer finns
implementerade i biblioteket \coq{}EAL\cite{coqeal}. Där finns också funktioner
som översätter mellan olika representationer av polynom och matriser.

Genom att sedan visa att samma resultat fås om man först applicerar vår redan
implementerade Toom-Cook-algoritm och sedan översätter resultatet till
polynomlistor som om man först översätter polynomen till listor och sedan
applicerer listversionen av Toom-Cook-algoritmen på dessa skulle man då kunna
säkerställa att listversionen ger korrekta resultat för de argument som är
intressanta, vilka är listor där det sista elementet är nollskiljt, som därmed
är listversioner av polynom.
