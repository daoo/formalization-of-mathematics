\section{Är beroendetypning användbart?}
Beroendetypning är bra, men innan man kan motivera beroendetypning behöver man
veta vad som är bra med ett starkt typsystem.
Typer specificerar restriktioner för vad kod kan göra, kompilatorn kan sedan se
till att dessa restriktioner följs. Det kan alltså ses som ett verktyg som
hjälper till att få koden korrekt. Eftersom typer är en slags dokumentation
tvingas också programmeraren att tänka på
\emph{vad} det är koden ska göra istället för \emph{hur} detta ska göras. Det
ses som positivt då många buggar uppkommer då kod skrivs utan att ha koll på
vad den ska göra. Denna motivering gäller generellt för typsystem och kan syfta
på till exempel \haskell{}. Men det gäller även för beroendetypade system då
beroendetypning är ett generellare system.

Beroendetypning gör att det ofta framgår exakt vad en funktion gör bara genom
att läsa typdeklarationen. Detta är en stor fördel när man söker efter en
specifik funktion som man inte vet namnetpå men man vet vilka typer den ska ha.
Att typerna ofta förklarar vad funktionerna gör att programmerare inte behöver
sätta sig in i avancerad kod i funktionen utan det räcker med att läsa vilka
typer funktionen har. Ett exempel på detta är bevis i \coq{} där typerna
specifierar vad som ska bevisas medan funktionskropparna ofta är väldigt
svårförstådda och inte säger så mycket om vad som bevisas.

Ett exempel är matriser med dimensionerna i typen som gör att man inte kan
tappa bort en dimension eller av misstag vända på dem. För matrismultiplikation
vet vi till exempel att om man multiplicerar en $m \times n$-matris och en
$n \times o$-matris får man en $m \times o$-matris. Med denna kunskapen kan
man definera multiplikations-rutinen i stil med:

\begin{verbatim}
matMul :: Matrix m n -> Matrix n o -> Matrix m o
matMul = ...
\end{verbatim}

Denna definitionen ger två fördelar:

\begin{itemize}
  \item Det går inte att skicka in fel sorts matriser. Den andra matrisen måste
    ha lika många rader som den första matrisen har kolumner, annars kompilerar
    koden inte.
  \item När man implementerar själva multiplikationen är man tvungen att få rätt
    dimensioner på resultatmatrisen.
\end{itemize}

Till skillnad från en definition med en lista av listor som i vår \haskell{} kod.
När man gör fel med den definitionen kraschar programmet vid körning med något
mindre hjälpsamt felmeddelande i stil med ''index out of bounds''. Eller så går
det ännu värre och funktionen kastar bort hälften av det tänkta resultatet.
Dessa problem kan man slippa när man använder beroendetypning.
