\section{Val av interpolationspunkter}
Interpolationspunkterna är en intressant del i algoritmen. Beroende på val av m
i Toom-Cook-m som väljs så avgörs också hur många punkter som behövs, nämligen
$2m-1$, för att entydigt bestämma koefficienterna i polynomet. Det förutsätter
att integritetsområdet är tillräckligt stort, eftersom punkterna måste vara
olika, och att punkterna också väljs så att interpolationsmatrisen blir
inverterbar. Om integritetsområdet innehåller få element så är en lösning för
att få tillgång till fler punkter att vi även tar punkter ur polynomringen.
Punkterna bör dock vara av så låg grad som möjligt eftersom att polynomen $u$ och
$v$ från algoritmen, se ekvationerna (\ref {eq:u}) och (\ref{eq:v}), helst ska
avta i grad snabbt för att få så få rekursiva anrop som möjligt, se
avsnitt~\ref{sec:rekursiv}, för att algoritmen ska vara effektiv. När polynomen
u och v evalueras i interpolationspunkterna så tillhör de evaluerade polynomen
$R[x]$, alltså
$u(\alpha_0),\dots,u(\alpha_{2m-2}),v(\alpha_0),\dots,v(\alpha_{2m-2}) \in
R[x]$. Graden av dessa polynom avgörs av graden av koefficienterna till $u$
respektive $v$, alltså $u_0,\dots,u_{m-1},v_0,\dots,v_{m-1} \in R[x]$,
tillsammans med graden som interpolationspunkterna bidrar med när polynomen
evalueras. Denna grad bör inte vara större än graden av ursprungspolynomen,
alltså de polynom som skulle multipliceras från början, för det tar bort lite
av poängen med algoritmen då idén med algoritmen är att dela upp polynomen så
graden avtar.

En annan lösning är att definiera polynomen $u$ och $v$ så att de beror på två
variabler istället för en. Denna definition är tagen ur \cite{bodrato2007notes}
och lyder
\begin{equation*}
  u(y,z) = \sum\limits_{i=0}^{m-1} {u_iz^{m-1-i}y^i}
\end{equation*}
där $u(x^b,1) = u$, analogt för v. Detta leder till att man evaluerar i $R$x$R$
eller i $R[x]$x$R[x]$ så att interpolationspunkterna är par av element i $R$
eller i $R[x]$. Till exempel evaluera i punkten $(0,1)$ i $u(y,z)$ ger $u_0$
vilket motsvarar den konstanta termen i $u(y)$ och evaluera $(1,0)$ i $u(y,z)$
ger $u_{n-1}$ som motsvarar den ledande koefficienten i u(y). Om en punkt
$(\alpha,\beta)$ väljs så får inte en multipel till denna punkten väljas,
alltså $(\lambda\alpha,\lambda\beta)$ där $\lambda$ är ett element i ringen
eller polynomringen, eftersom det ger upphov till en linjärt beroende
interpolationsmatris och då kan den inte inverteras.
