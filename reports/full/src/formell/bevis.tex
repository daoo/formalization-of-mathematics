\section{Formellt bevis av Toom-Cook}
Det här avsnittet presenterar det formella beviset för att den implementerade
algoritmen är korrekt. Många av detaljerna är rent tekniska och utelämnas. De
viktigaste taktikerna som används kommer att förklaras under bevisets gång.
Variablerna $p$ och $q$ kommer alltid beteckna polynom över ett
integritetsområde.

Beviset använder sig när så är möjligt av redan tidigare bevisade resultat från
Ssreflects bibliotek. En lång rad lemman om bland annat grundläggande
egenskaper hos polynom, naturliga tal, matriser och summor finns tillgängliga.

Det formella bevisets struktur är modellerad efter pappersbevisets, med ett
huvudbevis som bygger på ett flertal lemman. De formella lemmorna är dock många
fler än de för pappersbeviset, 15 jämfört med två. Det beror delvis på att
påståenden delats upp i flera delpåståenden för att underlätta
arbetsfördelningen inom gruppen, men i första hand på att resonemangssteg som i
pappersbeviset ses som så triviala att de inte ens nämns också måste visas när
beviset formaliseras. En ytterligare anledning är att det är blir mer
hanterbart att göra omskrivningar och deduktionssteg på mindre delpåståenden
separat. Dessutom används ett 50-tal mer generella lemman ur
SSREFLECT-biblioteket och ett par lemman som tillhandahållits av projektets
handledare, Anders Mörtberg.

Nedan beskrivs huvudbeviset i detalj, och därefter beskrivs några av de mer
intressanta delarna i bevisen av lemmorna.

\subsection{Huvudbeviset}
Detta är den formella motsvarigheten till prop \ref{prop:1} för den
implementerade algoritmen.
\begin{lstlisting}
Lemma toom_cook_correct : forall p q,
  toom_cook p q = p * q.
\end{lstlisting}
Eftersom \C{toom_cook} bara anropar \C{toom_cook_rec} med ett visst argument så
beror dess korrekthet helt på att \C{toom_cook_rec}, är korrekt:
\begin{lstlisting}
Lemma toom_cook_rec_correct : forall (n : nat) p q,
  unitmx V_I -> toom_cook_rec n p q = p * q.
\end{lstlisting}
Sedan är beviset för att själva algoritmen, där den rekursiva delen anropas med
maximum av graden av polynomen som argument, en direkt följd av det ovanstående
lemmat:
\begin{lstlisting}
Lemma toom_cook_correct : forall p q,
  toom_cook p q = p * q.
Proof. move=> p q. by apply: toom_cook_rec_correct. Qed.
\end{lstlisting}
Det formella beviset för att den rekursiva delen av algoritmen är korrekt har
en struktur som liknar beviset av prop \ref{prop:1}. Den största skillnaden att
induktionen i det formella fallet görs över antalet rekursiva anrop av
algoritmen i stället för att som i beviset av prop \ref{prop:1} över graden av
polynomen som multipliceras. Det beror på att den rekursiva delen av den
implementerade algoritmen är definierad med en DUMMYVARIABEL n, som beskrivits
i \ref{section:formrec}. I stället för att induktionen bygger på att graden av
de polynom som Toom-Cook anropas med minskar bygger den på att variabeln n
minskar. I båda fallen innebär dock induktionsantagandet att vi antar att
algoritmen fungerar korrekt för nästa rekursiva anrop. Den andra stora
skillnaden mellan pappersbeviset och det formella beviset är att många tekniska
detaljer, som kan tas för givna i ett informellt bevis måste visas explicit.
\begin{lstlisting}
Lemma toom_cook_rec_correct : forall (n : nat) p q,
  unitmx V_I -> toom_cook_rec n p q = p * q.
Proof.
elim=> [ // | n IHn p q V_inver] /=.
  set b := exponent m p q.
  set u := split m b p.
  set v := split m b q.
  case: ifP => [ // | _ ].
    * have ->:
      \col_i toom_cook_rec n ((evaluate u) i 0) ((evaluate v) i 0) =
      \col_i ((evaluate u) i 0 * (evaluate v) i 0).
      apply/colP => j.
      by rewrite mxE [X in _ = X]mxE (IH2 _ _ V_inver).
    rewrite 2!matrix_evaluation.
    have ->:
      \col_i ((\col_j u.[inter_points j 0]) i 0 *
      (\col_j v.[inter_points j 0]) i 0) =
      \col_i (u * v).[(inter_points i 0)].
      apply/colP => k.
      by rewrite 4!mxE -hornerM.
    rewrite toom_cook_interpol //; last by apply: size_split_mul.
    rewrite /recompose hornerM.
    2?recompose_split //.
    rewrite /b exponentC.
    by apply: exp_m_degree.
    by apply: exp_m_degree.
Qed.
\end{lstlisting}
Rad \C{4} anger med \C{elim=>} att beviset kommer göras med induktion över den
variabel som står först i lemmat, \C{n}. Basfallet är trivialt precis som i
prop \ref{prop:1} eftersom \C{toom_cook_rec} är vanlig multiplikation när n är
0 och delmålet det utgör avslutas direkt med hjälp av \C{//}. I
induktionssteget som börjar till höger om \C{|} instansieras variabeln \C{n'}
där \C{n' + 1}  = \C{n}, induktionsantagnandet \C{IHn}, polynomen \C{p} och
\C{q} och antagandet \C{V_inver} om att interpolationsmatrisen är inverterbar.
Till sist skriver \C{/=} om målet genom att utveckla \C{toom_cook_rec} till
funktionens definition.

Raderna \C{5} till \C{7} sätter kortare beteckningar på några av
uttrycken i målet för att öka bevisets läslighet. Då har målet och kontexten:
\begin{lstlisting}
n' : nat
IHn : forall p q, unitmx V_I -> toom_cook_rec n' p q = p * q
p : {poly R}
q : {poly R}
V_inver : unitmx V_I
b := exponent m p q : nat
u := split m b p : {poly {poly R}}
v := split m b q : {poly {poly R}}
______________________________________(1/1)
(if (size p <= 2) || (size q <= 2)
then p * q
else
recompose b
(interpolate
(\col_i
toom_cook_rec n' ((evaluate u) i 0) ((evaluate v) i 0)))) =
p * q
\end{lstlisting}
\C{if ... then ... else} - satsen säger att uttrycket i vänsterledet är lika
med \C{p * q} om graden av p eller q är mindre än 2 (eftersom multiplikation
i algoritmen då utförs direkt) och annars lika med det mer komplicerade
uttrycket som står efter \C{else}. Dessa två uttryck skulle motsvarat
basfallet och induktionssteget om induktionen hade gjort över graden av p och
q, som i prop \ref{prop:1}. Nu görs i stället på rad 6 med \C{ifP} en
falluppdelning över \C{if ... then ... else}-uttrycket som inte ger oss
något nytt induktionsantagende. I det första fallet när
\C{(size p <= 2) || (size q <= 2)} är höger- och vänsterled i målet trivialt
lika och löses liksom det triviala målet på rad \C{4} direkt med \C{//}.
När det är gjort är målet:
\begin{lstlisting}
recompose b (interpolate (\col_i toom_cook_rec n'
  ((evaluate u) i 0)
  ((evaluate v) i 0)))
  = p * q
\end{lstlisting}
och vi vill använda induktionsantagandet
\begin{lstlisting}
IHn : forall p q, unitmx V_I -> toom_cook_rec n' p q = p * q
\end{lstlisting}
och antagandet \C{V_inver} om att \C{V_I} är inverterbar för att skriva om
\begin{lstlisting}
toom_cook_rec n' ((evaluate u) i 0) ((evaluate v) i 0)
\end{lstlisting}
till \C{((evaluate u) i 0) * ((evaluate v) i 0)} inne i kolonnvektorn
\begin{lstlisting}
\col_i toom_cook_rec n' ((evaluate u) i 0)((evaluate v) i 0)
\end{lstlisting}
Eftersom definitionen \C{\\col_i} är låst för beräkning??? i Ssreflect och
uttrycket beror av indexet \C{i} i kolonnvektorn kan vi inte direkt skriva om
uttrycket, se AVSNITT DET OCH DET (DETTA KANSKE ÄR SANT). Så för att kunna göra
det öppnar vi ett nytt delmål med taktiken \C{have} på rad \C{9} där vi bara
arbetar med denna del av vänsterledet. Där använder vi på rad \C{12} lemmat
\C{colP} som låter oss visa att två vektorer är lika genom att visa att
elementen på motsvarande platser är lika i båda vektorerna. (Då vektorer är
implementerade i SSREFLECT som funktioner från mängden av index till mängden av
element i vektorn säger lemmat mer specifikt att två vektorer är lika om de
extensionellt sett är samma funktion, dvs om de antar samma värden för samma
argument.) På rad \C{13} kan vi sedan slutligen skriva om uttrycket med hjälp
av induktionshypotesen och målet har då blivit:
\begin{lstlisting}
recompose b (interpolate (\col_i
  ((evaluate u) i 0 * (evaluate v) i 0)))
  = p * q
\end{lstlisting}
Det återstår nu att visa tre saker: För det första att evalueringsfunktionen
faktiskt ger tillbaka en vektor att polynomens värden i
interpolationspunkterna, för det andra att interpolationsfunktionen givet dessa
vektorer ger oss koefficienterna i produktpolynomet och för det tredje att
recompose-funktionen och split-funktionen under vissa förutsättningar är
varandras inverser.

Det första och det andra är enkla följder av definitionen av
matrismultiplikation och invers och visas inte explicit i pappersbeviset. I det
formella beviset visas detta i två lemman, \C{matrix_evaluation} och
\C{toom_cook_interpol}, som används på rad \C{14} respektive rad \C{15} till
\C{20} för att skriva om målet i huvudbeviset till
\begin{lstlisting}
recompose b
(interpolate (\col_i (u * v).[inter_points i 0])) =
p * q
\end{lstlisting}
och sedan till
\begin{lstlisting}
recompose b (u * v) = p * q
\end{lstlisting}
För att matrismultiplikationen ska fungera Lemmat \C{toom_cook_interpol} har
som antagande att graden på det polynom som ska interpoleras, $u \cdot v$, är
mindre än antalet interpolationspunkter, så när det åberopas uppkommer ett till
delmål att visa:
\begin{lstlisting}
______________________________________(2/3)
size (u * v) <= number_points
\end{lstlisting}
Det görs på rad \C{20} genom lemmat \C{size_split_mul}. Detta är
ytterligare något som är självklart på pappret, eftersom det är så vi har
definiterat $u$ och $v$. Vi behöver också visa att interpolationsmatrisen
\C{V_I} är inverterbar, men eftersom det är ett av antagandena i kontexten
är det trivialt och visas på rad \C{20} med \C{//}.

Sedan utvecklar vi på rad \C{22} definitionen av \C{recompose} och använder
lemmat \C{hornerM} som säger att $(p \cdot q)(x) = p(x) \cdot q(x)$, för att
skriva om målet till:
\begin{lstlisting}
u.['X^b] * v.['X^b] = p * q
\end{lstlisting}
Då kan vi på rad \C{23} använda lemmat \C{recompose_split}, som motsvarar lemma
\ref{lemma:2} i pappersbeviset, för att visa att sammansättningen att de
uppdelade polynomen genom att evaluera i $x^b$ är korrekt och därmed skriva om
\C{u.['X^b]} till \C{p} och \C{v.['X^b]} till \C{q}. Då har vi visat att
vänsterledet i det ursprungliga målet är lika med högerledet \C{p * q}.

Det som återstår är sedan att bevisa motsvarigheten till Påstående 1 i lemma
\ref{lemma:2}, som gör att villkoren för att \C{recompose_split} ska kunna
användas på \C{p} och \C{q} är uppfyllda.
\begin{lstlisting}
______________________________________(1/2)
size q <= m * b
______________________________________(2/2)
size p <= m * b
\end{lstlisting}
Det görs på rad \C{25} till \C{26} med lemmat \C{exp_m_degree} efter att först
på rad \C{24} ha skrivit om det ena av målet med ett lemma för att funktionen
\C{exponent} är kommutativ, vilket avslutar det formella beviset för att Toom
Cook ger ett korrekt resultat.

\subsection{Lemman till huvudbeviset}
Det här stycket är lite ogenomtänkt än. Jag måste nog se det på papper och få
lite avstånd till det innan jag kommer på vad jag ska ta med. Förutom de 50-tal
lemman ur Ssreflectbiblioteken som har används i huvudbeviset har ett 15-tal
specifika lemman till huvudsatsen formulerats och bevisats. Här presenteras
några av dem. Det viktigaste lemmat är
\begin{lstlisting}
Lemma recompose_split : forall (f: {poly R}) (b: nat),
  size f <= m * b -> (split m b f).['X^b] = f.
\end{lstlisting}
som säger att \C{recompose} (som evaluerar ett polynom i $x^b$) är
vänsterinvers till \C{split} för polynom som har tillräckligt låg grad jämfört
med argumenten \C{m} och \C{b}. Lemmat motsvarar ungefär lemma \ref{lemma:2}
från efter Påstående 1. Beviset av \C{recompose_split} bygger på 5 mindre
lemman. \C{recompose_split_lemma1} används för att visa
\C{recompose_split_lemma2} som sedan används för att visa
\C{recompose_split_lemma3} som används för att visa huvudlemmat tillsammans med
motsvarigheten till Påstående 1, som visas separat i
\begin{lstlisting}
Lemma exp_m_degree_lemma : forall p,
  m > 0 -> size p <= m * succn (size p %/ m).
\end{lstlisting}
Det första lemmat är en specik instans av (\ref{eq:name8}), dvs av att
\begin{align}
  a(x)= a(x)  \modu x^b + \left(a(x)/x^b\right) x^b \label{eq:name8}
\end{align}
\begin{lstlisting}
Lemma recompose_split_lemma1 : forall (f: {poly R}) (k b: nat),
  (rmodp (rdivp f 'X^(k*b)) 'X^(b)) * 'X^(k*b) + (rdivp f 'X^(k.+1*b)) * 'X^(k.+1*b) =
  (rdivp f 'X^(k*b)) * 'X^(k*b).
\end{lstlisting}
Det andra lemmat motsvarar att (\ref{eq:name9}) = (\ref{eq:name10}), det vill
säga induktionssteget
i \ref{lemma:2}.
\begin{lstlisting}
Lemma recompose_split_lemma2 : forall (f: {poly R}) (k b: nat),
  \big[+%R/0]_(i < k.+1) ((rmodp (rdivp f 'X^(i*b)) 'X^b)*'X^(i*b)) +
  (rdivp f 'X^(k.+1*b))*'X^(k.+1*b) =
  \big[+%R/0]_(i < k) ((rmodp (rdivp f 'X^(i*b)) 'X^b)*'X^(i*b)) +
  (rdivp f 'X^(k*b))*'X^(k*b).
\end{lstlisting}
Det tredje lemmat svarar mot att (\ref{eq:name9}) = $p(x)$, så att givet
induktionssteget \C{recompose_split_lemma2} och basfallet
\C{recompose_split_lemma1} så gäller
\begin{lstlisting}
Lemma recompose_split_lemma3 : forall (f : {poly R}) (k b : nat),
  \big[+%R/0]_(i < k.+1) ((rmodp (rdivp f 'X^(i*b)) 'X^b)*'X^(i*b)) +
  (rdivp f 'X^(k.+1*b))*'X^(k.+1*b) = f.
\end{lstlisting}
Beviset av huvudsatsen byggde detta bygger mer på aritmetiska egenskaper,
egenskaper hos division hos polynom och egenskaper hos summor.

Precis som beviset av huvudsatsen för att visa likhet mellan vektorer får vi
här använda liknande trick för att visa likhet mellan summor genom att visa att
motsvarande term i respektive summa är lika.
\begin{lstlisting}
Lemma recompose_split_lemma0 p m n :
  rdivp p ('X^m * 'X^n) = rdivp (rdivp p 'X^m) 'X^n.
Proof.
\end{lstlisting}
