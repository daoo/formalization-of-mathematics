\documentclass[a4paper]{article}
\usepackage[swedish]{babel}
\usepackage[utf8]{inputenc}
\usepackage{a4wide}
\usepackage{fancyhdr}

\pagestyle{fancy}

\begin{document}
\author{Grupp 11}
\title{Formalisering av Algoritmer och Matematiska Bevis}
\date{\today}

\lhead{Formalisering av Algoritmer och Matematiska Bevis}
\rhead{Grupp 11}

\begin{abstract}
Todo
\end{abstract}

\maketitle
\thispagestyle{empty}
\newpage
\tableofcontents
\newpage

\section{Inledning}
Formalisering av matematik

\subsection{Proof Assistant}
\subsection{Coq}
\subsection{Toom-Cook}

\section{Metod}

Målet med vårt projekt är att bevisa en algoritm i Coq med hjälp av SSReflect.
För att uppnå detta behöver vi först lära oss Coq och SSReflect. Vårt projekt
kan därför delas in i tre delar: inlärning, test, samt implementation och
bevis.

\subsection{Inlärning}
Den första delen av vårt projekt går ut på att lära oss Coq och SSReflect. För
att göra detta använde vi oss i huvudsak av kursmaterial från andra
universitet, nämligen

\begin{itemize}
  \item Software Foundations av Benjamin. C. Pierce från University of
    Pennsylvania - En kursbok för Coq
  \item MAP Spring School organiserad av Inria - En introduktionskurs i
    SSReflect
\end{itemize}

\subsection{Test av Kunskaper}
Bevisa Karatsuba. Implementation i Haskell.

\subsection{Implementation och Bevis}

\section{Resultat}
\end{document}

% vim: set fdm=marker fmr=(fold),(end) :
