\documentclass[a4paper]{article}
\usepackage[fixlanguage]{babelbib}
\usepackage[swedish]{babel}
\usepackage[utf8]{inputenc}
\usepackage{a4wide}
\usepackage{amsmath}
\usepackage{authblk}
\usepackage{fancyhdr}
\usepackage[sort]{natbib}

\pagestyle{fancy}
\selectbiblanguage{swedish}

\title{Formalisering av Algoritmer och Matematiska Bevis}

\author[1]{Jesper Andersson}
\author[1]{Daniel Oom}
\author[1]{Niclas Ståhl}
\author[2]{Åsa Lideström}
\author[2]{Anders Sjöberg}
\affil[1]{Datateknik, Chalmers}
\affil[2]{Mattematik, Göteborgs Universitet}

\renewcommand\Authands{ och }

\date{\today}

\lhead{Formalisering av Algoritmer och Matematiska Bevis}
\rhead{Grupp 11}

\begin{document}
\begin{abstract}
Todo
\end{abstract}

\maketitle
\thispagestyle{empty}
\newpage
\tableofcontents
\newpage

\section{Inledning}
Formalisering av matematik

\subsection{Proof Assistant}
\subsection{Coq}
\subsection{Toom-Cook}

\section{Metod}

Målet med vårt projekt är att bevisa en algoritm i Coq med hjälp av SSReflect.
För att uppnå detta behöver vi först lära oss Coq och SSReflect. Vårt projekt
kan därför delas in i tre delar: inlärning, test, samt implementation och
bevis.

\subsection{Inlärning}
Den första delen av vårt projekt går ut på att lära oss Coq och SSReflect. För
att göra detta använde vi oss av följande kursmaterial från andra universitet:
\begin{itemize}
  \item Software Foundations av Benjamin. C. Pierce från University of
    Pennsylvania
  \item MAP Spring School organiserad av Inria
\end{itemize}
Samt ett antal artiklar:
\begin{itemize}
  \item Coq in a Hurry
  \item Coq Master
  \item SSReflect Tutorial
\end{itemize}
\citet{bertot2006coq}

\subsection{Test av Kunskaper}
Bevisa Karatsuba. Implementation i Haskell.

\subsection{Implementation och Bevis}

\section{Resultat}

\bibliography{report}
\bibliographystyle{alpha}
\nocite{*}
\end{document}

% vim: set fdm=marker fmr=(fold),(end) :
