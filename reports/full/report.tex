\documentclass[a4paper]{article}
\usepackage[fixlanguage]{babelbib}
\usepackage[numbers,sort]{natbib}
\usepackage[swedish]{babel}
\usepackage[utf8]{inputenc}
\usepackage{a4wide}
\usepackage{amsmath}
\usepackage{authblk}
\usepackage{fancyhdr}

\pagestyle{fancy}
\selectbiblanguage{swedish}

\title{Formalisering av Algoritmer och Matematiska Bevis}

\author[1]{Jesper Andersson}
\author[1]{Daniel Oom}
\author[1]{Niclas Ståhl}
\author[2]{Åsa Lideström}
\author[2]{Anders Sjöberg}
\affil[1]{Datateknik, Chalmers}
\affil[2]{Mattematik, Göteborgs Universitet}

\renewcommand\Authands{ och }

\date{\today}

\lhead{Formalisering av Algoritmer och Matematiska Bevis}
\rhead{Grupp 11}

\begin{document}
\begin{abstract}
Todo
\end{abstract}

\maketitle
\thispagestyle{empty}
\newpage
\tableofcontents
\newpage

\section{Inledning}
Formalisering av matematik

\subsection{Proof Assistant}
\subsection{Coq}
\subsection{Toom-Cook}

\section{Metod}

Målet med vårt projekt är att bevisa en algoritm i Coq med hjälp av SSReflect.
För att uppnå detta behöver vi först lära oss Coq och SSReflect. Vårt projekt
kan därför delas in i tre delar: inlärning, test, samt implementation och
bevis.

\subsection{Inlärning}
Den första delen av vårt projekt gick ut på att lära oss Coq och SSReflect. För
detta krävdes utbilndningsmaterial och eftersom bevisassistenter var helt nytt
för oss visste vi inte vad vi skulle leta efter. Men vår handledare hjälpte oss
att komma igång med kursmaterial från andra universitet och ett par artiklar:
\begin{itemize}
  \item Software Foundations
  \item Coq in a Hurry
  \item Coq Master
\end{itemize}
Detta material innhåller övningar som vi studerade huvudsaklingen enskilt och
sedan jämförde våra lösningar i gruppmöten.

När vi fått en grundläggande förståelse av Coq började vi ge oss in på
SSReflect. Samma arbetsgång användes som för Coq, övningar och inläsning fast
med ett annat, SSReflect-inriktat, material:
\begin{itemize}
  \item MAP Spring School organiserad av Inria
  \item SSReflect Tutorial
\end{itemize}
I detta skede började vi även också titta på och försöka förstå Toom-Cook.

\subsection{Test av Kunskaper}
Nästa del i arbetet var ett mer praktiskt test av vad vi lyckats lära oss.
Detta genom att bevisa en enklare algorithm, Karatsuba, i Coq och SSReflect.
Karatsuba är ett specialfall av Toom-Cook och vår handledare hade redan gjort
ett bevis och föreslog att vi också skulle göra det.

Bevisa Karatsuba. Implementation i Haskell.

\subsection{Implementation och Bevis}

\section{Resultat}

\bibliographystyle{plainnat}
\bibliography{report}
\nocite{*}
\end{document}

% vim: set fdm=marker fmr=(fold),(end) :
